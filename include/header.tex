% Source: http://home.sandiego.edu/~dhoffoss/latex/latex-paper-template.tex

\setlength{\textwidth}{17.2cm}     % if you change this, consider changing
\setlength{\evensidemargin}{-.3cm} % side margins to retain centering
\setlength{\oddsidemargin}{-.3cm}

\setlength{\textheight}{23cm}   % if you change this, consider changing
\setlength{\topmargin}{-2cm}  % top margin to retain centering
\setlength{\headsep}{1.6cm}

%---------------------- These packages below add functionality to your version of LaTeX --------------
%---------------------- You might not use all of them --------------------------------------
\usepackage{amssymb}
\usepackage{latexsym}
\usepackage{amsthm}
\usepackage{enumerate}
\usepackage{epsfig}
\usepackage{graphicx}
\usepackage{color}
\usepackage{float}
\usepackage{subfigure}
\usepackage{amsmath}
\usepackage{makeidx}
\usepackage{fancyhdr}
\pagestyle{fancy}
\usepackage{lastpage}
\usepackage{url}


%------------------------- Customized Header --------------------------------------------------------
\fancyhead{}
\fancyfoot{}			
\lhead{Team \# 123}
\rhead{Page \thepage\ of \pageref{LastPage}}

%---------- the symbols below will give you the blackboard bold of R, T, etc. ----------
\DeclareSymbolFont{AMSb}{U}{msb}{m}{n}  
\DeclareMathSymbol{\Sph}{\mathbin}{AMSb}{"53} 
\DeclareMathSymbol{\R}{\mathbin}{AMSb}{"52}
\DeclareMathSymbol{\T}{\mathbin}{AMSb}{"54} 
\DeclareMathSymbol{\Z}{\mathbin}{AMSb}{"5A}
\DeclareMathSymbol{\K}{\mathbin}{AMSb}{"4B}

%------------------------- Theorem and Proof Environments -------------------------------------------------

% This section defines all the environments you might use.  Just type
% \begin{theorem, or corollary, or whatever}, then the optional name of the
% theorem inside {} (or empty {} if no name), then body of the theorem,
% corollary, whatever, also inside {} then \end{theorem, corollary, whatever}
%
% Notice when I use them in the paper, I put an optional "argument" to the function
% and this gives a name to the theorem

\newenvironment{theorem}[1]{\vspace{.9cm}\noindent    {\bf Theorem {#1}}}{\vspace{.1cm}}
\newenvironment{lemma}[1]{\vspace{.9cm}\noindent    {\bf Lemma {#1}}}{\vspace{.1cm}}
\newenvironment{corollary}[1]{\vspace{.9cm}\noindent    {\bf Corollary {#1}}}{\vspace{.1cm}}
\newenvironment{definition}{\vspace{.9cm}\noindent {\bf Definition}}{\vspace{.1cm}}
\def\qed{\hfill $\Box$}
\renewenvironment{proof}{\vspace{.5cm}   \noindent{\bf Proof: }}{\qed \vspace{1cm}}
 
%\theoremstyle{definition}
%\newtheorem{notation}[theorem]{Notation}
%\newtheorem{properties}[theorem]{Properties}
%\newtheorem{remark}[theorem]{Remark}
%\newtheorem{example}[theorem]{Example}
%\newtheorem{claim}[theorem]{Claim}
%\newtheorem{observation}[theorem]{Observation}
%\newtheorem{definition}[theorem]{Definition}


% ---------------------- Define case environment ------------------------------

\newcounter{case}

\newenvironment{case}[1]{\stepcounter{case} \addvspace{.5\baselineskip} \noindent\textbf{Case \thecase}. \textit{#1}}{\hfill\fbox{Case \thecase}}

%\newtheorem{case}{Case}
\newtheorem{subcase}{Case}[case]
\newtheorem{sub2case}{Case}[subcase]
%\newtheorem{sub3case}{Case}[sub2case]
%\newtheorem{sub4case}{Case}[sub3case]


%Picture inclusion

\newcommand\pic[3]{
\begin{figure}[H] \begin{center} 
\epsfig{file=#1, height=#2pt} 
\end{center} 
\caption{#3} 
\end{figure}
}

\def\inj{\text{inj}}
\def\diam{\text{diam}}
\def\area{\text{area}}
\def\length{\text{length}}

