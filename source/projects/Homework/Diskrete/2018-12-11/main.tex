\documentclass[12pt]{article}   % you have 10pt, 11pt, or 12pt options

\usepackage{import}
\subimport{../}{setup.tex}

\title{Diskrete Matehmatik}
\author{Stefan Kreiner}
\date{2018-12-11}


\begin{document}

\maketitle

\begin{figure}[ht]
\centerline{\hspace{-1cm}
    \includegraphics[height={6cm}]{resources/31.png}} 
\end{figure}

a) Die Anzahl aller verschiedenen Telefonnummern ist die Anzahl aller Permutationen von 7 Elementen ohne die Permutationen von nicht unterscheidbaren Objekten.
$=>$ Die Anzahl aller verschiedenen Telefonnummern ist $7!/(3!*2!)$ = 420


b) Mit 2 beginnern alle Telefonnummern der form 2xxxxxx wobei xxxxxx eine permutation von 1,2,2,5,8,8 ist
$=>$ Es gibt wegen gleicher Begründung wie oben $6!/(2!*2!)$ = 180 verschiedene Telefonnummern die mit 2 beginnen.

Analog für 5 Telefonnummern: $6!/(3!*2!) = 60$
($=>$ es gibt auch 60 Telefonnummern die mit 1 beginnen)


c) es gibt 60 die mit 1 beginnen und 180 die mit 2 beginnen
$=>$ Die kleinste Telefonnummer die mit 5 beginnt ist an stelle 241.

Position 400:
Es gibt 60 Telefonnummern mit 1 \& 5 am Anfang und 180 mit 2 am Anfang 
$=>$ Es gibt 300 Telefonnummern mit 1,2 oder 5 am Anfang

für 81xxxxx gibt es $5!/3! = 20$ verschiedene Möglichkeiten
für 82xxxxx gibt es $5!/2! = 60$ verschiedene Möglichkeiten
für 85xxxxx gibt es $5!/3! = 20$ verschiedene Möglichkeiten
$=>$ an Position 400 steht die größte Telefonnummer die mit 85xxxxx gemacht werden kann
$=>$ an Position 400 steht 8582221


\begin{figure}[ht]
\centerline{\hspace{-1cm}
    \includegraphics[height={5cm}]{resources/32.png}} 
\label{fig:fun}
\end{figure}


\end{document}
