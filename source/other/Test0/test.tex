\documentclass[a4paper,twoside]{book}

\usepackage{Project0/preamble}

\begin{document}
\section{Kompaktheit}

Wir haben gesehen, daß stetige Bilder offener Mengen nicht offen, und stetige Bilder
abgeschlossener Mengen nicht abgeschlossen zu sein brauchen. In den nächsten
zwei Paragraphen werden wir Eigenschaften topologischer Räume untersuchen,
welche unter stetigen Abbildungen erhalten bleiben. Die dabei gewonnenen Erkenntnisse
sind von weitreichender Bedeutung und werden uns insbesondere beim
Studium reellwertiger Funktionen äußerst nützlich sein.
\medskip
\subsubsection{Die Überdeckungseigenschaft}
\smallskip
Im folgenden bezeichnen wir mit $X \coloneqq (X, d)$ stets einen metrischen Raum.
\smallskip

Ein Mengensystem  $\{ A_\alpha \subset X  \sep \alpha \in \A \}$ heißt \textbf{Überdeckung} der Teilmenge 
$K \subset X$, falls es zu jedem $x \in K$ ein $\alpha \in \A$ gibt mit $x \in A_\alpha$, d.h., falls $K \subset \bigcup_\alpha A_\alpha$. 
Eine Überdeckung heißt \textbf{offen}, wenn jedes $A_\alpha$ offen ist in $X$. Schließlich heißt eine 
Teilmenge $K \subset X$ \textbf{kompakt}, falls jede offene Überdeckung von $K$ eine \textit{endliche}
Teilüberdeckung besitzt.

\subsection{Beispiele} \myenum{(a)} Es sei $(x_k)$ eine konvergente Folge in $X$ mit Grenzwert $a$. Dann
ist die Menge $K \coloneqq \{a\} \cup \{x_k \sep k \in \N\}$ kompakt. 

\begin{Beweis0}
Es sei $\{ O_\alpha \sep \alpha \in \A\}$ eine offene Überdeckung von $K$. Dann gibt es Indizes $\alpha$ und $\alpha_k \in \A$ mit $\alpha \in O_\alpha$ und $x_k \in O_{\alpha,k}$ für $k \in \N$. 
Ferner gibt es wegen $\lim x_k = a$ ein $N \in \N$ mit $x_k \in O_{\alpha,k}$ für $k > N$. 
Also ist $\{ O_{\alpha,k} \sep 0 \le k \le N\} \cup \{O_\alpha\}$ eine endliche Teilüberdeckung der gegebenen Überdeckung von $K$. \qed
\end{Beweis0}

\noindent\myenum{(b)} Die Aussage von $(a)$ ist i. allg. falsch, wenn der Grenzwert $a$ aus $K$ entfernt wird.

\begin{Beweis0} Es seien $X \coloneqq \R$ und $A \coloneqq \{ 1/k \sep k \in \N^× \}$. Ferner setzen wir $O_1 \coloneqq (1/2, 2)$ und
$O_k \coloneqq (1/(k + 1), 1/(k - 1))$ für $k \ge 2$. Dann ist $\{ O_k \sep k \in \N^x\}$ eine offene Überdeckung von $A$ mit der Eigenschaft, daß jedes $O_k$ genau ein Element von $A$ enthält. 
Somit kann $\{ O_k \sep k \in \N\}$ keine endliche Teilüberdeckung von A besitzen. \qed\smallskip
\end{Beweis0} 

\noindent\myenum{(c)}  Die Menge der natürlichen Zahlen $\N$ ist nicht kompakt in $\R$.

\begin{Beweis0} 
Es genügt wieder, eine offene Überdeckung $\{ O_k \sep k \in \N\}$ von $\N$ anzugeben, 
bei der jedes $O_k$ genau eine natürliche Zahl enthält, z.B. $O_k \coloneqq (k - 1/3, k + 1/3)$ für $k \in N$. \qed\smallskip
\end{Beweis0}

\subsection{Satz}  \textit{Jede kompakte Menge $K \subset X$ ist abgeschlossen und beschränkt in $X$.}\smallskip

\begin{Beweis}
Es sei $K \subset X$ kompakt.\smallskip

(i) Wir beweisen zuerst die Abgeschlossenheit von $K$ in $X$. Offensichtlich
genügt es, den Fall $K \neq X$ zu betrachten, da $X$ abgeschlossen in $X$ ist. Es sei
also $x_0 \in K^c$. Wegen der Hausdorffeigenschaft gibt es zu jedem $y \in K$ offene Umgebungen $U_y \in \mathfrak{U}(y)$ und $Vy \in \mathfrak{U}(x_0)$ mit $U_y \cap V_y = \emptyset$. Da $\{ U_y ; y \in K \}$ eine offene
Überdeckung von $K$ ist, finden wir endlich viele Punkte $y_0, . . . , y_m$ in $K$ mit
$K \subset \bigcup^m_{j=0} U_{y_j} \eqqcolon U$. Dann sind $U$ und $V \coloneqq \bigcap^m_{j=0} V_{y_j}$ wegen Satz 2.4 offen und
disjunkt. Also ist $V$ eine Umgebung von $x_0$ mit $V \subset K^c$, d.h., $x_0$ ist innerer Punkt
von $K^c$. Da dies für jedes $x_0 \in K^c$ gilt, ist $K^c$ offen. Also ist $K$ abgeschlossen.\smallskip

(ii) Um die Beschränktheit von $K$ zu verifizieren, fixieren wir ein $x_0$ in $X$.
Wegen $K \subset \bigcup^\infty_{k=1} \B(x_0, k) = X$, da $\B(x_0, k)$ gemäß Beispiel 2.2 offen ist, und wegen
der Kompaktheit von $K$ finden wir $k_0, . . . , k_m \in \N$ mit $K \subset \bigcup^m_{j=0} \B(x_0, k_j)$. Somit
gilt $K \subset \B(x_0,N)$, wobei wir $N := max\{k_0, . . . , k_m\}$ gesetzt haben. Also ist $K$
beschränkt. \qed
\end{Beweis}

\subsubsection{Eine Charakterisierung kompakter Mengen}

Die Umkehrung von Satz 3.2 ist in allgemeinen metrischen Räumen falsch.\footnote{Vgl. Aufgabe 15.} Kompakte
Mengen können i. allg. \textit{nicht} als abgeschlossene und beschränkte Mengen
charakterisiert werden. Hingegen gelingt in metrischen Räumen folgende Kennzeichnung
kompakter Mengen, in deren Beweis wir die nachstehende Definition
verwenden: Eine Teilmenge $K$ von $X$ heißt \textbf{totalbeschränkt}, wenn es zu jedem
$r > 0$ ein $m \in \N$ und $x_0, . . . , x_m \in K$ gibt mit $K \subset \bigcup^m_{k=0} \B(x_k, r)$. Offensichtlich
ist jede totalbeschränkte Menge beschränkt.

\subsection{Theorem} \textit{Eine Teilmenge $K \subset X$ ist genau dann kompakt, wenn jede Folge
in $K$ einen Häufungspunkt in $K$ besitzt}. \medskip

\begin{Beweis} (i) Zuerst nehmen wir an, daß $K$ kompakt sei, und daß es in $K$ eine
Folge gebe ohne Häufungspunkt in $K$. Zu jedem $x \in K$ finden wir dann eine
offene Umgebung $U_x$ von $x$ mit der Eigenschaft, daß $U_x$ höchstens endlich viele
Folgenglieder enthält. Weil $\{ U_x \sep x \in K \}$ natürlich eine offene Überdeckung
von $K$ ist, gibt es $x_0, . . . , x_m \in K$, so daß $K$ von $\{ U_{x_k} \sep k = 0, . . .,m\}$ überdeckt
wird. Somit kann $K$ nur endlich viele Folgenglieder enthalten, und folglich keinen
Häufungspunkt besitzen. Dieser Widerspruch zeigt, daß jede Folge in $K$ einen
Häufungspunkt in $K$ hat.
\smallskip

(ii) Den Beweis der umgekehrten Implikation führen wir in zwei Teilschritten.\smallskip

(a) Es sei $K$ eine Teilmenge von $X$ mit der Eigenschaft, daß jede Folge in $K$
einen Häufungspunkt in $K$ besitzt. Wir behaupten: $K$ ist totalbeschränkt.
\smallskip

Wir führen wieder einen Widerspruchsbeweis. Nehmen wir also an, daß $K$
nicht totalbeschränkt sei. Dann gibt es ein $r > 0$ und ein $x_0 \in K$, so daß $K$ nicht
in $\B(x_0, r)$ enthalten ist. Insbesondere finden wir ein $x_1 \in K \setminus \B(x_0, r)$. Aus demselben
Grund gibt es auch ein $x_2 \in K \setminus [\B(x_0, r) \cup \B(x_1, r)]$. 
Nach diesem Verfahren fortschreitend finden wir induktiv eine Folge $(x_k)$ in $K$ mit der Eigenschaft, daß
$x_{n+1}$ nicht zu $\bigcup^n_{k=0}\B(x_k, r)$ gehört. Denn andernfalls gäbe es ein $n_0 \in \N$, so daß
$K$ in $\bigcup^n_{k=0} \B(x_k, r)$ enthalten wäre, was die Totalbeschränktheit von K bedeuten
würde. Gemäß unserer Voraussetzung besitzt die Folge $(x_k)$ einen Häufungspunkt
$x$ in $K$. Somit gibt es $m,N \in \N^×$ mit $d(x_N, x) < r/2$ und $d(x_{N+m}, x) < r/2$.
Aus der Dreiecksungleichung folgt dann $d(x_N, x_{N+m}) < r$, d.h., $x_{N+m}$ gehört
zu $\B(x_N, r)$, was nach Konstruktion der Folge $(x_k)$ nicht möglich ist. Dies beweist,
daß $K$ totalbeschränkt ist.
\smallskip

(b) Es sei nun $\{O_\alpha \sep \alpha \in A\}$ eine offene Überdeckung von $K$. Da $K$ totalbeschränkt ist, gibt es zu jedem $k \in \N^×$ endlich viele offene Bälle mit Radien $1/k$ und Mittelpunkten in $K$, welche $K$ überdecken. Nehmen wir an, es gäbe keine
endliche Teilüberdeckung von $\{O_\alpha \sep \alpha \in A\}$ für $K$. Dann finden wir zu jedem
$k \in \N^×$ einen dieser endlich vielen offenen Bälle mit Radius $1/k$, er heiße $B_k$,
so daß $K \cap B_k$ nicht durch endlich viele der $O_\alpha$ überdeckt wird. Bezeichnen wir
mit $x_k$ den Mittelpunkt von $B_k$ für $k \in \N^×$ , so ist $(x_k)$ eine Folge in $K$, die gemäß
Voraussetzung einen Häufungspunkt $\bar{x}$ in $K$ besitzt. Es sei nun $\bar{\alpha} \in A$ ein Index mit
$\bar{x} \in O_{\bar{\alpha}}$. Da $O_{\bar{\alpha}}$ offen ist, gibt es ein $\varepsilon > 0$ mit $\B(\bar{x}, \varepsilon) \subset O_{\bar{\alpha}}$. Andererseits finden
wir ein $M >2/\varepsilon$ mit $d(x_M, \bar{x}) < \varepsilon/2$, da $\bar{x}$ ein Häufungspunkt von der Folge $(x_k)$
ist. Folglich gilt für jedes $x \in B_M$ die Abschätzung
\begin{equation*}
d(x, \bar{x}) \le d(x, x_M) + d(x_M, x) < \frac{1}{M}+\frac{\varepsilon}{2}<\frac{\varepsilon}{2}+\frac{\varepsilon}{2} = \varepsilon,
\end{equation*}
d.h., es gelten die Inklusionen $B_M \subset \B(\bar{x}, \varepsilon) \subset O_{\bar{\alpha}}$. Dies ist aber nach Konstruktion
von $B_M$ nicht möglich. Also besitzt $\{O_\alpha \sep \alpha \in \A\}$ eine endliche Teilüberdeckung.\qed
\end{Beweis}

\subsubsection{Folgenkompaktheit}
Eine Teilmenge $A \subset X$ heißt \textbf{folgenkompakt}, wenn jede Folge in $A$ eine in $A$ konvergente
Teilfolge besitzt. Die Charakterisierung von Häufungspunkten einer Folge
durch konvergente Teilfolgen aus Satz II.1.17 und das eben bewiesene Theorem ergeben
unmittelbar:

\subsection{Theorem} \textit{Eine Teilmenge eines metrischen Raumes ist genau dann kompakt,
wenn sie folgenkompakt ist.} \\


Als weitere wichtige Anwendung von Theorem 3.3 können wir kompakte Teilmengen
der Räume $\K^n$ kennzeichnen.

\subsection{Theorem} (Heine-Borel) \textit{Eine Teilmenge von $\K^n$ ist genau dann kompakt,
wenn sie abgeschlossen und beschränkt ist.}\\

\textit{Insbesondere ist ein Intervall in R genau dann kompakt, wenn es abgeschlossen
und beschränkt ist.}

\begin{Beweis} Gemäß Satz 3.2 ist jede kompakte Menge abgeschlossen und beschränkt.
Die umgekehrte Aussage ergibt sich aus dem Satz von Bolzano-Weierstraß (vgl.
Theorem II.5.8), Satz 2.11 und Theorem 3.4. \qed
\end{Beweis}

\subsubsection{Stetige Abbildungen auf kompakten Räumen}

Das folgende Theorem zeigt, daß die Kompaktheit unter stetigen Abbildungen
erhalten bleibt.

\subsection{Theorem} \textit{Es seien $X, Y$ metrische Räume, und $f : X \rightarrow Y$ sei stetig. Ist $X$
kompakt, so ist auch $f(X)$ kompakt}, d.h., stetige Bilder kompakter Mengen sind
kompakt. \smallskip

\begin{Beweis} Es sei $\{O_\alpha \sep \alpha \in \A\}$ eine offene Überdeckung von $f(X)$ in $Y$. Aufgrund
von Theorem 2.20 ist dann $f^{-1}(O_\alpha)$ für jedes $\alpha \in \A$ eine offene Teilmenge von $X$.
Somit ist $\{ f^{-1}(O_\alpha) \sep \alpha \in \A\}$ 
eine offene Überdeckung des kompakten Raumes $X$.
Deshalb finden wir endlich viele Indizes $\alpha_0, . . . , \alpha_m \in \A$ mit $X = \bigcup^m_{k=0} f^{-1}(O_{\alpha_k} )$.
Folglich gilt $f(X) \subset \bigcup^m_{k=0} O_{\alpha_k}$, 
d.h., $\{O_{\alpha_0}, . . .,O_{\alpha_m}\}$ ist eine endliche Teilüberdeckung
von $\{O_\alpha \sep \alpha \in \A\}$ für $f(X)$. Also ist $f(X)$ kompakt. \qed
\end{Beweis}\smallskip

\subsection{Korollar} \textit{Es seien $X, Y$ metrische Räume, und $f : X \rightarrow Y$ sei stetig. Ist $X$
kompakt, so ist $f(X)$ beschränkt.}\smallskip

\begin{Beweis} 
Dies folgt sofort aus Theorem 3.6 und Satz 3.2. \qed
\end{Beweis}\smallskip

\subsubsection{Der Satz vom Minimum und Maximum}

Für reellwertige Funktionen ergibt sich aus Theorem 3.6 der überaus wichtige Satz,
daß jede reellwertige stetige Funktion auf kompakten Mengen ihr Minimum und
ihr Maximum annimmt.

\subsection{Korollar} (Satz vom Minimum und Maximum) \textit{Es sei $X$ ein kompakter metrischer
Raum, und $f : X \rightarrow R$ sei stetig. Dann nimmt die Funktion $f$ in $X$ ihr
Minimum und ihr Maximum an, d.h., es gibt $x_0, x_1 \in X$ mit} \medskip
\begin{equation*}
  f(x_0) = \underset{x \in X}{\min} f(x) \hspace{5mm} \text{und} \hspace{5mm} f(x_1) = \underset{x \in X}{\max} f(x).
\end{equation*}
\medskip

\begin{Beweis} Aus Theorem 3.6 und Satz 3.2 wissen wir, daß $f(X)$ abgeschlossen und beschränkt ist in $\R$. 
Deshalb existieren $m \coloneqq \inf(f(X))$ und $M \coloneqq \sup(f(X))$ in $\R$.
Wegen Satz I.10.5 gibt es Folgen $(y_n)$ und $(z_n)$ in $f(X)$, die in $\R$ gegen $m$ bzw. $M$
konvergieren. Da $f(X)$ abgeschlossen ist, folgt aus Satz 2.11, daß $m$ und $M$ zu $f(X)$
gehören. Somit gibt es $x0, x1 \in X$ mit $f(x_0) = m$ und $f(x_1) = M$. \qed
\end{Beweis}

Die Bedeutung das Satzes vom Minimum und Maximum wollen wir durch
die nachstehenden wichtigen Anwendungen beleuchten.

\subsection{Beispiele} \myenum{(a)} Auf $\K^n$ sind alle Normen äquivalent.

\begin{Beweis0} (i) Es bezeichnen $\enorm{\!\cdot\!}$ die euklidische und $\norm{\cdot}$ eine beliebige Norm auf $\K^n$. Dann
genügt es, die Äquivalenz dieser zwei Normen, d.h. die Existenz einer positiven Konstanten
$C$ mit
\begin{equation}
C^{-1} \enorm{x} \le \norm{x} \le C \enorm{x}, \hspace{1cm} x \in \K^n ,
\end{equation}
nachzuweisen. \smallskip

(ii) Wir setzen $S \coloneqq \{ x \in \K^n \sep \enorm{x} = 1\}$. Aus Beispiel 1.3(j) wissen wir, daß die
Funktion $\enorm{\cdot} : \K^n \rightarrow \R$ stetig ist. Somit folgt aus Beispiel 2.22(a), daß $S$ in $\K^n$ abgeschlossen
ist. Natürlich ist $S$ auch beschränkt in $\K^n$. Wir können also den Satz von
Heine-Borel anwenden und erkennen $S$ als kompakte Teilmenge von $\K^n$.\smallskip

(iii) Nun zeigen wir, daß $f : S \rightarrow R, x \mapsto \norm{x}$ stetig ist.\footnote{Der Leser mache sich klar, daß die Aussage von Beispiel 1.3(j) hier nicht anwendbar ist!} 
Dazu sei $e_k$, $1 \le k \le n$, die Standardbasis in $\K^n$. 
Für jedes $x = (x_1, . . . ,x_n) \in \K^n$ gilt dann $x = \sum^n_{k=1} x_ke_k$ (vgl.
Beispiel I.12.4(a) und Bemerkung I.12.5). Somit folgt aus der Dreiecksungleichung für $\norm{\cdot}$ die Abschätzung
\begin{equation}
\norm{x} = \norm{\sum\limits_{k=1}^n x_ke_k} 
\le \sum\limits_{k=1}^n \enorm{x_k} \norm{e_k} \le C_0 \enorm{x}, \hspace{1cm} x \in \K^n,
\end{equation}
wobei wir $C_0 \coloneqq \sum_{k=1}^n \norm{e_k}$ gesetzt und die Ungleichungen 
$\enorm{x_k} \le \enorm{x}$ verwendet haben.
Dies beweist bereits das zweite Ungleichheitszeichen von (3.1). Ferner folgt aus (3.2) und
der umgekehrten Dreiecksungleichung für $\norm{\cdot}$ die Abschätzung
\begin{equation*}
|f(x) - f(y)| = \enorm{\norm{x}-\norm{y}} \le \norm{x - y} \le C_0 \enorm{x - y} , \hspace{1cm} x,y \in S ,
\end{equation*}
was die Lipschitz-Stetigkeit von $f$ zeigt. \smallskip

(iv) Für jedes $x \in S$ gilt $f(x) > 0$. Deshalb folgt aus dem Satz vom Minimum, daß
$m \coloneqq \min f(S)$ positiv ist, d.h., wir erhalten die Abschätzung
\begin{equation}
0 < m = \min f(S) \le f(x) = \left\lVert x \right\rVert , \hspace{1cm} x \in S .
\end{equation}

\noindent Schließlich sei $x \in \K^n\setminus\{0\}$. Dann gehört $x/|x|$ zu $S$, und aus (3.3) folgt somit 
$m \le \left\lVert x / |x| \right\rVert$.
Also gilt
\begin{equation}
m|x| \le \left\lVert x \right\rVert , \hspace{1cm} x \in \K^n .
\end{equation}
Die Behauptung ergibt sich nun aus (3.2) und (3.4) mit $C \coloneqq max\{C_0, 1/m\}$. \qed
\end{Beweis0}\smallskip

\noindent\myenum{(b) Fundamentalsatz der Algebra\footnote{Der Fundamentalsatz der Algebra gilt nicht über dem Körper der reellen Zahlen R, wie das
Beispiel $p = 1+X2$ zeigt.}} \textit{Jedes nichtkonstante Polynom} $p \in \mathbb{C}[X]$ \textit{besitzt eine Nullstelle in} $\mathbb{C}$\textit{.}

\begin{Beweis} (i) Es sei $p$ ein solches Polynom. Wir schreiben $p$ in der Form
\begin{equation*}
p = X^n + a_{n-1}X^{n-1} + \cdot\cdot\cdot + a_1X + a_0
\end{equation*}
\end{Beweis}
\end{document}
