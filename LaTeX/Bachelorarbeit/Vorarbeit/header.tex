
\usepackage{ifthen}

% Mathematik
\usepackage{amsthm}
\usepackage{amsmath}

\newcounter{counter}[section]
\newenvironment{template}[2]{
	\refstepcounter{counter}
  	\vspace{.9cm}
  	\noindent    

  	\textbf {
	    \thesection.\thecounter \quad #1  \ifthenelse{\equal{#2}{}}{}{({#2})}
	    \newline
  	}
  	\begingroup\itshape
} {
	\endgroup
  	\par\bigskip  
}
\newenvironment{theorem}[1][]{
	\begin{template}{Satz}{#1}
} {
	\end{template}
} 
\newenvironment{lemma}[1][]{
	\begin{template}{Lemma}{#1}
} {
	\end{template}
} 
\newenvironment{corollary}[1][]{
	\begin{template}{Korrolar}{#1}
} {
	\end{template}
} 
\newenvironment{definition}[1][]{
	\begin{template}{Definition}{#1}
} {
	\end{template}
} 
\newenvironment{proposition}[1][]{
	\begin{template}{Proposition}{#1}
} {
	\end{template}
} 
\newenvironment{remark}{
	\textbf{Remark: }
} {
}
