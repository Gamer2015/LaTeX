\documentclass[a4paper,10pt]{article}

\title{Exponent of SL(2, p) and SL(2, 2^r)}
\author{Stefan Kreiner}
\date{\today}


% Mathematik
\usepackage{ifthen}
\usepackage{amsmath}

\usepackage{mdframed}

\newcounter{counter}[section]
\newenvironment{template}[2]{
	\refstepcounter{counter}
  	\noindent    

  	\textbf {
	    \thesection.\thecounter \quad #1  \ifthenelse{\equal{#2}{}}{}{({#2})}
	    \newline
	    \newline
  	}
  	\begingroup\itshape
} {
	\endgroup
}
\newenvironment{theorem}[1][]{
	\begin{template}{Theorem}{#1}
} {
	\end{template}
} 
\newenvironment{lemma}[1][]{
	\begin{template}{Lemma}{#1}
} {
	\end{template}
} 
\newenvironment{corollary}[1][]{
	\begin{template}{Corollary}{#1}
} {
	\end{template}
} 
\newenvironment{definition}[1][]{
	\begin{template}{Definition}{#1}
} {
	\end{template}
} 
\newenvironment{proposition}[1][]{
	\begin{template}{Proposition}{#1}
} {
	\end{template}
} 
\newenvironment{proof}{
	\textit{Proof:}
} {
	
} 
\newenvironment{remark}{
	\textbf{Remark: }
} {
}

\usepackage[ngerman]{babel}
\usepackage[T1]{fontenc}
\usepackage[applemac]{inputenc}
\usepackage{parskip}

\usepackage{amsmath}
\usepackage{amssymb}

\usepackage{enumerate}

\bibliographystyle{unsrt}

\begin{document}

\tableofcontents

\section{Exponent of SL(2, p)}

\begin{lemma} \label{GeneralizedQuaternionGroupNonCyclicForNontrivialCase}
	Let Q be a generalized quaternion group of order $2^r$ with $|Q| >= 8$.

	Then Q is not cyclic.
\end{lemma}

\begin{proof}
	Q has order $2^r$ and is generated by two elements $h$ and $k$ such that
	\begin{enumerate}
		\item The order of $h$ is $2^{r-1}$ \label{GeneralizedQuatternionGroupAxiom1}
		\item $k^2 = h^{2^{r-2}}$ \label{GeneralizedQuatternionGroupAxiom2}
		\item $hk = kh^{-1}$ $^{\cite[p. 29]{gorenstein2007finite}}$ \label{GeneralizedQuatternionGroupAxiom3}
	\end{enumerate}
	Using \ref{GeneralizedQuatternionGroupAxiom1} and \ref{GeneralizedQuatternionGroupAxiom2}, all elements $w \in Q$ can then be written in the form $w = k^jh^i$ with $j \in \{0,1\}$. Elements of the form $w = kh^i$ have order 4 because
	\begin{equation*}
		w^2 = kh^ikh^i = kkh^{-i}h^i = k^2.
	\end{equation*}
	We now know the orders of all elements in $Q$ and there is no element with order $2^r$, therefore the group is not cyclic.
\end{proof}

\begin{theorem} \label{ExponentOfSL2p}
	Let p be an odd prime, $G := SL(2, p)$ be the special linear group over the field with p Elements.

	Then $exp(G) = \frac{|G|}{2}$.
\end{theorem}

\begin{proof}
	It is sufficient to show that (a) for all odd primes $q$ there is a cyclic Sylow $q$-subgroup of $G$, (b) a Sylow $2$-subgroup of G is not cyclic and (c) there exists an element of order $2^{r-1}$, where $2^r$ $\Vert$ $|G|$.

	We know $|G| = (p^2-1)p = p(p-1)(p+1)$, $G$ contains cyclic subgroups with order $p-1$ and $p+1$ $^{\cite[p. 40-42]{gorenstein2007finite}}$ and G contains a subgroup of order $p$ because of the first Sylow theorem and this group must be cyclic because $p$ is prime.

	(a) Let $q$ be an odd prime. If $q$ does not divide $|G|$ the Sylow $q$-subgroup is trivial. If $q$ does divide $|G|$ then $q$ divides $p$, $p-1$ or $p+1$. As $q$ is an odd prime, and therefore larger than 2, $q$ can only divide one of the those three factors. 

	Let $m \in \{ p, p-1, p+1\}$ be the factor such that $q$ divides $m$ and let $r$ be the positive integer such that $q^r$ $\Vert$ $|G|$. Because of the choice of $m$ there is a cyclic subgroup $C$ of $G$ with order $m$ and because $q$ does not divide any of the other two factors of $G$ we have $q^r$ $\Vert$ $m$. With the first Sylow theorem $C$ has a Sylow $q$-subgroup $S$ with order $q^r$. $S$ is now a Sylow $q$-subgroup of $G$ because $q^r$ $\Vert$ $|G|$ and $S$ is cyclic because $C$ is cyclic. 

	(b, c) We know that the Sylow $2$-subgroups of $G$ are isomorphic to a generalized quaternion group because $p$ is odd$^{\cite[p. 42]{gorenstein2007finite}}$. Let $S$ be a Sylow $2$-subgroup of $G$ and let $r$ be the positive integer such that $|S| = 2^r$ $\Vert$ $|G|$. With $p$ being odd, 2 divides both $p-1$ and $p+1$ and one of $p-1$ and $p+1$ is a multiple of 4, therefore $2^3$ $\vert$ $|S|$ and $|S| >= 8$. By Lemma \ref{GeneralizedQuaternionGroupNonCyclicForNontrivialCase} we know that S is not cyclic and we have proven (b). The order of S is $2^r$, hence we get an element we need for (c) from the definition of generalized quaternion groups.
\end{proof}

\section{Exponent of SL(2, $2^r$)}

\begin{theorem}
	Let r be a positive natural number, $G := SL(2, 2^r)$ be the special linear group over the field with $2^r$ Elements.

	Then $exp(G) = 2((2^r)^2 - 1)$.
\end{theorem}

\begin{proof}
	We know $|G| = 2^r((2^r)^2-1) = 2^r(2^r-1)(2^r+1)$ and $G$ contains cyclic subgroups with order $2^r-1$ and $2^r+1$$^{\cite[p. 40-42]{gorenstein2007finite}}$.

	All odd prime factors of $|G|$ are in the factor $(2^r)^2-1$ and 2 divides |G|. Therefore G has an element of order 2 and it is sufficient to show that (a) for all odd primes $q$ there is a cyclic Sylow $q$-subgroups of $G$ and (b) all nontrivial elements in a Sylow 2-subgroup have order 2.

	(a) Let $q$ be an odd prime. If $q$ does not divide $|G|$ the Sylow $q$-subgroup is trivial. If $q$ does divide $|G|$ then $q$ divides $2^r$, $2^r-1$ or $2^r+1$. As $q$ is an odd prime it cannot divide $2^r$ and it is larger than 2, therefore $q$ can only divide either $2^r-1$ or $2^r+1$. We can now make the same argument as in the proof of Theorem \ref{ExponentOfSL2p} to conclude that there is a Sylow $q$-subgroup of $G$.

	(b) We know that a Sylow $2$-subgroup of $G$ is elementary abelian because $2^r$ is even$^{\cite[p. 42]{gorenstein2007finite}}$, therefore all nontrivial elements in a Sylow $2$-subgroup of $G$ have order 2, concluding the proof.
\end{proof}



\bibliography{references}

\end{document}