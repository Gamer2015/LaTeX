%Vorlage für akademische Arbeiten
%erstellt von Mag. Martin Glatz
%mathematik@oehunigraz.at
%mathematik.oehunigraz.at
%ig-mathe (Studienvertretung Mathematik Uni Graz)
%Stand: März 2015

% *****************************************************************
% % ********************** Präambel *******************************
% % ***************************************************************


%===================================================================
%Dokumentklasse...
%===================================================================
\documentclass[
		fontsize=10pt,		%Standardschriftgröße auf 10 pt gesetzt
		a4paper,			%Papierformat: b=148mm h=210mm
		parskip=half-,		%kein Absatzeinzug, stattdessen Höhe einer halben Zeile.
		twoside,			%zweiseitiges Layout (Seitenzahlen außen)
		headings=big,	%große überschriften samt abstände
		appendixprefix=true,
		]{scrreprt}		%koma-script: europäische Formatvorlage für  scrreprt (chapter-ebene vorhanden)



%===================================================================
%Standard-Codierungen und Schriften...
%===================================================================

\usepackage[utf8]{inputenc}	%Zeichencodierung in den Textfiles
\usepackage[english,ngerman]{babel}	%deutsche Rechtschreibung, Worttrennung
\usepackage[T1]{fontenc} 		%Schriftcodierung
\usepackage{lmodern}			%LaTeX-Schriftpaket laden (lmodern)
%\usepackage{microtype} % für zeilenumbrüche etc, verträgt sich nicht mit  allen schriften. Einfach probieren oder in der dokumentation nachlesen


%Pakete




%======================================================
%Standard-Mathe-Pakete
%======================================================

\usepackage{amsmath}
\usepackage{amsfonts}
\usepackage{amssymb}
\usepackage{amsthm}		%theorem-umgebungen etc
\usepackage{polynom}	%für polynomdivisonen

\usepackage{textcomp}		%zusätzliche symbole


%======================================================
%rmfamily-Schriften mit Mathe-Unterstützung
%======================================================


%es folgt eine auflistung von schriften, wenn man mit der latex-standardschrift lmodern für die rmfamily (serifenschrift) nicht zufrieden ist. die jeweils zuletzt geladen schrift wird aktiviert). einfach enstprechendes prozent-zeichen der gewünschten schrift entfernen

%\usepackage[sc]{mathpazo} % palatino mit mathe-schrift, recht breit

%\usepackage{mathptmx}		%klassische times mit mathe-schrift

%\usepackage{concrete}	%recht altbacken, schreibmaschinenähnlich: concrete

%\usepackage[charter]{mathdesign}	%schrift charter mit matheschrift.	

%\usepackage{fourier}  	%utopia mit fourier als matheschrift
%\usepackage[utopia]{mathdesign}		%wie fourier, etwas mehr abstand als fourier in mathe-formeln

%\usepackage{fouriernc}	%schoolbook mit fourier als mathe-schrift, recht altmodisch und niedrig

%========abgestimmtes charter-paket=============
%%siehe microtype-doku: pdf: http://mirror.unl.edu/ctan/macros/latex/contrib/microtype/microtype.pdf
%%und source-code: http://mirrors.xmu.edu.cn/CTAN/macros/latex/contrib/microtype/microtype.dtx
%\usepackage[charter]{mathdesign}	%charter mit matheschrift
% \def\rmdefault{bch} % not scaled
%\usepackage{sourcesanspro}	%sanserif-schrift passend für charter 
%\makeatletter
%\renewcommand{\SourceSansPro@scale}{1.02}
%\makeatother
%========ende abgestimmtes charter-paket=============
  


%======================================================
%sf-family-Schriften: 
%======================================================

%eine auswahl an sanserif-schriften. Wer mit der latex-Standardschrift bei der sfamily nicht zufrieden ist, entfernt einfach das Prozentzeichen bei der gewünschten schrift.

%\usepackage[math]{kurier} %angenehme schrift auch für fließtext mit matheschrift

%\usepackage[scaled]{helvet}	%helvetica, recht breit, kombinierbar mit euler-matheschrift
%\usepackage{euler}		%mathe-schrift zu helvetica

%\usepackage[scaled]{berasans}	%bitstream vera sans, eher breit

%\usepackage[default]{sourcesanspro}	%auch für fließtext, fett ist sehr aufdringlich

%\usepackage{DejaVuSansCondensed} %relativ kompakt

%\usepackage{libertine} %relativ verschnörkeld, schön für überschriften



%======================================================
%%fehlerhaft:
%======================================================

%\usepackage[bitstream-charter]{mathdesign}		%evtl fehler?
%\usepackage[urw-garamond]{mathdesign}		%geht nicht
%\usepackage{MinionPro}	% geht nicht...
%\usepackage{mathmanx}	%geht nicht
%\usepackage{pandora}		% geht nicht



%======================================================
%Hauptschriftfamily auswählen:
%======================================================

%\renewcommand{\familydefault}{\sfdefault}		%sans serif als Standard-Schrift, wenn gewünscht


%======================================================
%Schriften und physiaklische Einheiten
%======================================================

\usepackage[
	detect-all,		%alle umgebungseinstellungen der schriften werden auf die Befehle \SI, \num, \si übertragen. Wenn nicht gewünscht: auskommentieren.
	decimalsymbol=comma,
	exponent-product=\cdot,   %Voreinstellung: \times (also z.B. 2 \times 10^3)
	quotient-mode = fraction, %\num{2 / 3} wird als \frac{2}{3} übersetzt
	%per-mode=fraction,		%in Einheiten Brüche
	]{siunitx}



%\AtBeginDocument{\sisetup{detect-family=false, math-rm=\mathrm, text-rm=\rmfamily}}			%sinnvolle option, wenn fließtext in sffamily ist, die mathe schrift aber nicht. mit dieser option werden einheiten und zahlen der Befehle \SI, \si, \num automatisch der mathematik-schrift angepasst







%======================================================
%Textsatz-Pakete
%======================================================

\usepackage[autostyle,german=guillemets]{csquotes} %nachträgliche umgestaltung von anführungszeichen
\usepackage{paralist} %für compactitem
\usepackage{enumitem} % für [label={\alph*)}] oder [label={\roman*)}] usw. bei normalen umgebungen

\usepackage{setspace} %für Befehl \onehalfspacing
\onehalfspacing %für 1,5-facher Zeilenabstand. 



%======================================================
%Grafiken-Pakete
%======================================================

\usepackage{graphicx}			%einbinden von grafiken
\usepackage[x11names, table]{xcolor}		%ermöglichen von farben
\usepackage{pdfpages} 		%einbinden ganzer pdf-seiten (z.B für Anhang etc)


%tikz und pgf:
\usepackage{pgf}
\usepackage{pgfplots}
\pgfplotsset{compat=1.8}
\usepackage{mathtools}
\usepackage{tikz}
\usetikzlibrary{arrows}	% bei darf weitere bibliotheken laden. einfach  googeln...

\usepackage[framemethod=TikZ]{mdframed}			%Boxen mit Rahmen und Füllfarben

%======================================================
%ChehmiePakete
%======================================================

\usepackage{chemfig}		%strukturformeln zeichen

\renewcommand*\printatom[1]{\ensuremath{\mathsf{#1}}} %schrift umstellen auf sanserif in strukturformeln
%\setchemformula{format=\sffamily}
\setdoublesep{0.25700 em}  % 'Bond Spacing'
\setatomsep{2 em}%{1.78500 em}    % 'Fixed Length'
\setbondoffset{0.18265 em} % 'Margin Width'
\newcommand{\bondwidth}{0.09 em}%{0.06642 em} % 'Line Width'
\setbondstyle{line width = \bondwidth}

\usepackage[version=3]{mhchem}	%für chemische summenformeln


%======================================================
%Tabellen-Pakete
%======================================================

\usepackage{tabularx} 	%tabellen mit fester breite-> tabularx-Umgebung
\usepackage{multirow} 	%verschmelzen von Zellen mittels multirow
\usepackage{array}		%erweiterte funktionen bei TAbellen >{} oder @{}
\usepackage{booktabs}		%befehle für abstände und Linien: \midrule etc


%======================================================
%Pakete für Inhaltsverzeichnisse und Einstellungen...
%======================================================

\usepackage{tocloft}		%Foramtierung Abbildung und Tabellenverzeichnis
\usepackage{interfaces}


%=======================================
%Pakete zur Umschaltung von pdf-Digital-Version und Papierversion (größere Rand innen wegen Bindung)
%=======================================

\usepackage{ifthen}		%erlaubt if-Abfrage (für Druck-Web-Umschaltung): druck: png-grafiken, web:pdf-grafiken

\newboolean{boolweb}
\setboolean{boolweb}{true}		%web-version eingeschalten: true; druck einschalten: false

%Syntax für boolweb:
%	\ifthenelse{\boolean{boolweb}}
%	{dann} 					%web= true-Anweisungen
%	{ansonsten}				%web=false (also druck)-Anweiseungen







\usepackage[
	colorinlistoftodos,
	textwidth=3.5cm,
	textsize=footnotesize]{todonotes} %für notizen im pdf-Dokument



%===================================================================
%Einbinden der anderen Präambel-Dateien
%===================================================================



%design-einstellungen

%==============================================================================
%Farbdefinitionen: (bei Bedarf ändern) Hinweis: Beim Ausdrucken sind die Farben meist etwas dunkler als am Bildschirm (druckerabhängig)
%==============================================================================

\definecolor{igblaud}{RGB}{0,85,212}
\definecolor{blau1}{RGB}{0, 101, 186}

%Hellblau:
\definecolor{igblauh}{RGB}{42,127,255}
\definecolor{blockbg}{RGB}{214, 228, 255}
\definecolor{igblaubg}{RGB}{214, 228, 255}

\definecolor{igoranged}{RGB}{255,121,13}
\definecolor{igorange}{RGB}{255, 216, 77}
\definecolor{igorangehell}{RGB}{255, 213, 142}

\definecolor{hellgrau}{cmyk}{0, 0, 0, 0.4}

%farbzuweisungen:
\colorlet{farbe1}{blau1}
\colorlet{farbe2}{igblauh}
\colorlet{farbe3}{igorange}
\colorlet{farbe4}{igorangehell}

%==============================================================================
%Überschriftengestaltungen: Je nach Lust und Laune ändern
%==============================================================================

\addtokomafont{sectioning}{\sffamily}

\addtokomafont{chapter}{\mdseries}
\addtokomafont{part}{\color{farbe1}}
\addtokomafont{partentry}{\color{farbe1}}
\addtokomafont{chapter}{\color{farbe1}}
\addtokomafont{section}{\mdseries\color{farbe1}}
\addtokomafont{subsection}{\mdseries\color{farbe2}}


%\renewcommand*{\chapterheadstartvskip}{\vspace*{0mm}} %umdefinieren des abstands vor der chapter-überschrift, wenn gewünscht.

%==============================================================================
%Seitenränder
%==============================================================================

\usepackage[a4paper]{geometry}

\geometry{
		%twoside, %auskommentiert = einseitiges Layout
		bottom=25mm, %abstand text - rand
		top = 30mm, %abstand rand - text
		headsep = 1cm} %abstand kopfzeile - text
		

%bleibt für text: 210-60 = 150 mm

%Ränderanpassung für pdf-Version und Druckversion

\ifthenelse{\boolean{boolweb}}
{\geometry{inner=30mm,
		outer=30mm %symmetrische Ränder in digitalversion
		}} 
{\geometry{inner=35mm, 
		outer=25mm %innerer rand größer wegen bindung
		}}

%seitenabschluss unten nicht zwingend durch leerraumaufüllung auf gleicher höhe:
\raggedbottom

%==============================================================================
%Kopf und Fußzeile -- allgemein
%==============================================================================

\usepackage{fancyhdr}		%Paket für Kopf- und Fußzeile


\pagestyle{fancy}
\fancyhf{} 				%alle Kopf- und Fußzeilenfelder bereinigen

%Felder befüllen:
\fancyhead[R]{\small\color{hellgrau}\thepage}					%Seitenanzahl oben rechts 

%Formatierung Kolumnentitel
\renewcommand{\chaptermark}[1]{%
 \markboth{%
Kap. \thechapter.\ #1
 }{}}% vgl. Kursfolien Design, Seite 19.


\fancyhead[L]{\small\color{hellgrau}\nouppercase\leftmark}		%kapitelname: links oben
%ersetze \leftmark durch \rightmark, um die aktuelle section zu bekommen...


%Linienbreiten:
\renewcommand{\footrulewidth}{0pt} 					% keine Trennlinie unten
\renewcommand{\headrulewidth}{1.5pt} 				% obere Trennlinie
%}


%Umgestaltung Kopfzeilentrennlinie
\renewcommand\headrule
  {\color{hellgrau!60}%										%Farbzuweisung der Linie
  \vspace{2pt}												%vertikaler Abstand
  \hrule height\headrulewidth width\headwidth		%Linie (Höhe, Breite) zeichnen
 }


%Umgestaltung plain-Style (wird z.B. bei Chapter usw. automatisch aufgerufen

\fancypagestyle{plain}{%
\fancyhf{} 					%Felder bereinigen
\fancyhead[R]{\small\color{hellgrau}\thepage}		%Seitenzahl, oben, außen
}



%==============================================================================
%Fußnotengestaltungen
%==============================================================================


\usepackage[bottom, splitrule]{footmisc}
\newcommand{\mfootnote}[1]{\footnote{#1}$^)$}	%meine eigener fußnoten befehl. Verwenden statt \foonote{...}. 
\deffootnote[0mm]{0mm}{0mm}{\textbf{\thefootnotemark)} } %Fußnoten beginnen mit fetter, normaler zahl und Klammer. Wenn nicht gewünscht, dann % davor


%==============================================================================
%Beschriftungen von Abbildungen und Tabellen
%==============================================================================

\usepackage[labelfont=bf]{caption} %fette captions (z.B. Tab. xxx)

\captionsetup[figure]{name=Abb.}
\captionsetup[table]{name=Tab.}

%==============================================================================
%aufzählungen etc
%==============================================================================

%itemize-symbol verschiedener ebenen: quadrat in farbe
\renewcommand{\labelitemi}{\color{farbe2}\rule{1.3ex}{1.3ex}}
\renewcommand{\labelitemii}{\color{farbe3}\rule{1.3ex}{1.3ex}}
\renewcommand{\labelitemiii}{\color{farbe2}\rule{1ex}{1ex}}
\renewcommand{\labelitemiv}{\textbullet}


\setdefaultenum						%Nummerierungssymbole (enumerate etc) ändern
{\color{farbe1}\bfseries i)}			%Ebene 1,
{\color{black}\bfseries a)}			%Ebene 2
{\color{farbe2}\bfseries i.}			%Ebene 3
{\color{black}\bfseries A.}			%Ebene 4



%==============================================================================
%design-befehle
%==============================================================================

\newcommand{\ausz}[1]{\textit{#1}}	%bei bedarf anpassen...
	

%tabellenbefehle:

%========================================================
%Linien und Zeilenumbrüche
%========================================================

\newcommand{\nz}{\tabularnewline} %NeueZeile
\newcommand{\nzl}{\tabularnewline\midrule}%NeueZeileLinie
\newcommand{\nzlf}{\tabularnewline\midrule[1.5pt]} %NeueZeileLinieFett

%========================================================
%Verschmelzen von Zellen
%========================================================

%vertikale Zentrierung  ber zwei Zeilen in Tabellen: \zentr{inhalt}
\newcommand{\zentr}[1]{\multirow{2}{*}{#1}}	

% zentrieren innerhalb der spalten \zentr{anz}{inhalt}
\newcommand{\zentrn}[2]{\multirow{#1}{*}{#2}} 


%========================================================
%Begrenzungslinien
%========================================================

\newcommand{\obl}{\toprule[1.5pt]}%Obere Begrenzungslinie

\newcommand{\ubl}{\tabularnewline\bottomrule[1.5pt]}%UntereBegrenzungsLnie


%========================================================
%Sonstiges
%========================================================

\newcommand{\abst}{\hspace*{1cm}}
\newcommand{\pf}{$\rightarrow$}

\newcommand{\siehehinten}{\bigskip \hfill \pf}





%befehle Mathe




%===================================================================
%Befehle für Mengen
%===================================================================

\newcommand{\mN}{\mathbb{N}}
\newcommand{\mZ}{\mathbb{Z}}
\newcommand{\mQ}{\mathbb{Q}}
\newcommand{\mR}{\mathbb{R}}
\newcommand{\mC}{\mathbb{C}}

\renewcommand{\Re}{\mathrm{Re}}
\renewcommand{\Im}{\mathrm{Im}}

\newcommand{\ee}{\mathrm{e}} %eulersche Zahl
\newcommand{\ii}{\mathrm{i}} %imaginäre Einheit

\newcommand{\mK}{\mathbb{K}}	%körper
\newcommand{\mD}{\mathcal{D}} %definitionsbereich
\newcommand{\mG}{\mathbb{G}} %grundmenge
\newcommand{\mL}{\mathbb{L}} %lösungsmenge


%===================================================================
%Befehle für Normen/Metriken
%===================================================================

\newcommand{\absolut}[1]{\left\lvert#1\right\rvert}				%\betrag{zahl}
\newcommand{\norm}[1]{\left\|#1\right\|}						%\norm{vektor}
\newcommand{\ip}[2]{\left\langle #1,#2\right\rangle}			%\ip{v1}{v2}


%===================================================================
%Befehle für Ableitungen etc
%===================================================================


\newcommand{\pdiff}[2]{\frac{\partial #1}{\partial #2}}			%\pdiff{f (x_1,x_2)}{x_1}
\newcommand{\hpdiff}[3]{\frac{\partial^{#1} #2}{\partial #3^{#1}}}	%\hpdiff{3}{f(x_1,x_2)}{x_1}
\newcommand{\diff}[2]{\frac{d #1}{d #2}}						%\diff{f(x)}{x}


%===================================================================
%Befehle fürs Integrieren
%===================================================================

\newcommand{\ints}{\int \limits}							%\ints_{unten}^{oben}
\newcommand{\dd}[1]{\:\mathrm{d}{#1}}						%\dd{integrationsvariable}


%===================================================================
%Spezielle Brüche
%===================================================================

\newcommand{\pivier}{\frac{\pi}{4}}							
\newcommand{\pihalb}{\frac{\pi}{2}}
\newcommand{\oneover}[1]{\frac{1}{#1}}						%\oneover{nenner}


%===================================================================
%Befehle für Vektoren
%===================================================================


\newcommand{\vb}[1]{\mathbf{#1}}							%\vb{v} (vector-bold)
\newcommand{\svek}[1]{\begin{pmatrix} #1 \end{pmatrix}}	 %Spaltenvektor		%\svek{x_1 \\ x_2 \\ x_3}
\newcommand{\zvek}[1]{(#1)^T}		 %Spaltenvektor in Zeile geschreiben 				%\zvek{x_1,x_2,x_3}


%===================================================================
%Sonstiges
%===================================================================

\newcommand{\landau}[1]{\mathcal{O}(#1)}						%\landau{argument}
\newcommand{\Mul}{\cdotp}									%a \Mul b
\newcommand{\sums}{\sum\limits}							%\sums_{j=1}^{n}
\newcommand{\prods}{\prod\limits}							%\prods_{j=1}^{n}
\newcommand{\lims}{\lim\limits}							%\lims_{x \to \infty$
\DeclareMathOperator{\sgn}{sgn}


%eigene umgebungsbefehle


%===================================================================
%Bsb-Umgebungen mit mdframed
%===================================================================


\newmdenv[
	roundcorner=10pt,
	innerlinewidth=0pt,
	outerlinewidth=2pt, 
	outerlinecolor=farbe2, 
	innerlinecolor=white,
	middlelinewidth=0pt,
	%backgroundcolor=white, 
	innerleftmargin=0.5cm, 
	innerrightmargin=0.5cm, 
	innertopmargin=0.5cm, 
	innerbottommargin=0.3cm, %0.5cm, 
	skipabove =0,%.5\baselineskip , 
	skipbelow =0.25\baselineskip,
	splittopskip=5mm,
	splitbottomskip=3mm
]{boxblau}


%achtugn: mit der Verwendung von parskip=half- kann  es  probleme mit den abständen geben. man muss evtl. bei den abstände vor und nach den boxen etwas herumprobieren



%===================================================================
%mathematische Umgebungen mit mdframed (entweder-oder-Alternative zu ams theorem) die anderen Umgebungen auskommentieren
%===================================================================

\mdtheorem[
	roundcorner=10pt,
	innerlinewidth=0pt,
	outerlinewidth=2pt, 
	outerlinecolor=farbe2, 
	%innerlinecolor=white,
	middlelinewidth=0pt,
	%backgroundcolor=white, 
	innerleftmargin=16pt, 
	innerrightmargin=16pt, 
	innertopmargin=2mm, 
	innerbottommargin=0.3cm, %0.5cm, 
	skipabove =0,%.5\baselineskip , 
	skipbelow =0.25\baselineskip,
	splittopskip=5mm,
	splitbottomskip=3mm,
	%frametitlefont=\normalfont\mdseries
]{bsp}{\color{farbe2}Bsp.}[chapter] %optionales argument Chapter beginnt bei jedem kapitel mit Zählung bei 1: Bsp 1.1, Bsp 1.2, ... Bsp 2.1, Bsp 2.2 usw. (so findet man ein beispiel schneller...

\mdtheorem[
	outerlinewidth=0pt,
	innerlinewidth=0pt,
	middlelinecolor=hellgrau,
	middlelinewidth=0.75pt,
	backgroundcolor=hellgrau!20, 
	innerleftmargin=10pt, 
	innerrightmargin=10pt, 
	innertopmargin=2mm, 
	innerbottommargin=0.3cm, 
	skipabove =0,
	skipbelow =0.25\baselineskip,
	roundcorner=10pt,
	splittopskip=5mm,
	splitbottomskip=3mm,
	%frametitlefont=\normalfont\mdseries\color{red},
	%frametitlebackgroundcolor=black!30,
]{definition}{Definition}[chapter]

\mdtheorem[
	roundcorner=10pt,
	innerlinewidth=0pt,
	outerlinewidth=2pt, 
	outerlinecolor=farbe3, 
	%innerlinecolor=white,
	middlelinewidth=0pt,
	%backgroundcolor=white, 
	innerleftmargin=10pt, 
	innerrightmargin=10pt, 
	innertopmargin=2mm, 
	innerbottommargin=0.3cm, %0.5cm, 
	skipabove =0,%.5\baselineskip , 
	skipbelow =0.25\baselineskip,
	splittopskip=5mm,
	splitbottomskip=3mm,
	%frametitlefont=\normalfont\mdseries\color{red},
	%frametitlebackgroundcolor=black!30,
]{satz}{Satz}[chapter]

\newmdenv[
	frametitle={Beweis},
	hidealllines=true,
	leftline=true,
	innerlinewidth=0pt,
	outerlinewidth=2pt, 
	outerlinecolor=farbe3, 
	%innerlinecolor=white,
	middlelinewidth=0pt,
	%backgroundcolor=white, 
	innerleftmargin=10pt, 
	innerrightmargin=0pt, 
	innertopmargin=2mm, 
	innerbottommargin=0.3cm, %0.5cm, 
	skipabove =0,%.5\baselineskip , 
	skipbelow =0.25\baselineskip,
	splittopskip=5mm,
	splitbottomskip=3mm,
	%frametitlefont=\normalfont\mdseries\color{red},
	%frametitlebackgroundcolor=black!30,
		]{beweis}


%Beweissymbole und deren Einsatz... 

\newcommand{\beweisendeinzeile}{\hfill $\square$} %beweisende in der momendanenzeile.
\newcommand{\beweisende}{\vspace*{-0.5\baselineskip}\hfill $\square$} %beweisende
\newcommand{\beweisendemathe}{\vspace*{-1.5\baselineskip}\hfill $\square$} %beweisende nach abgesetzter mathe-formel


%===================================================================
%mathematische Umgebungen mit ams theorem (entweder-oder-Alternative zu mdframed) die anderen Umgebungen auskommentieren
%===================================================================

%
%\theoremstyle{plain}
%\newtheorem{satz}{Satz}[chapter]
%
%\theoremstyle{definition}
%\newtheorem{definition}{Definition}[chapter]
%\newtheorem{bsp}{Bsp}[chapter]
%
%
%\newenvironment{beweis}{\begin{proof}}{\end{proof}}
%%\qedhere %beweissymbol, sollte aber automatisch gemacht werden
%%\mbox{\qedhere} %wenn bei qedhere fehler kommt
%







%===================================================================
%Setup für Verlinkungen und Farben etc.
%===================================================================
\usepackage[
	linktocpage=true, 
	colorlinks=true,
	linkcolor=farbe1,
	citecolor=farbe1,
	urlcolor=farbe1
	]{hyperref}


%===================================================================
%Einstellungen Literatur: Momentan Numerisch. Bei Bedarf ändern
%===================================================================

\usepackage
	[
	backend=biber,		%		sortierprogramm
	%backend=bibtex, %alternatives sortierprogramm. hilfsdaten vor umstellung löschen
	style=numeric-comp,
	maxbibnames=99,
	maxcitenames=1,
	urldate=long,
	sorting=none
	]
	{biblatex}
\addbibresource{literatur-da.bib} %laden der literaturliste
\DefineBibliographyStrings{ngerman}{andothers={et\addabbrvspace al\adddot}}
\DefineBibliographyStrings{ngerman}{urlseen={abgerufen am}}




% *****************************************************************
% % ********************** Präambel *******************************
% % ***************************************************************



\begin{document}

\pagenumbering{Roman}


%===================================================================
%Beginn
%===================================================================


%Datei:	Titel.tex

%Part:	Beginn
%Chapter: 	Titel



\thispagestyle{empty}


\begin{center}

\vspace*{0.5cm}

\begin{huge}
\color{blau1}

Langtitel der Diplomarbeit evtl. über mehrere Zeilen, falls der Titel so lang ist


\end{huge}

\vspace*{3mm}

\Large{Untertitel der Diplomarbeit, falls vorhanden}

\vspace*{1.5cm}

\huge{Bachelorarbeit}

\vspace*{1cm}

\large

zur Erlangung des akademischen Grades \\
eines Bachelor der Naturwissenschaften

\vspace*{5mm}

an der Karl-Franzens-Universität Graz

\vspace*{2.5cm}


vorgelegt von


\Large{Stefan KREINER}

\vspace*{1cm}

am Institut für Mathematik und Wissenschaftliches Rechnen

Begutachter: Titel Vorname Nachname

\vspace*{1.5cm}

Graz, 2020

\includegraphics[width=2.5cm]{grafiken/uni-graz-logo.pdf} 

\end{center}

\clearpage %titelblatt
\clearpage

%Datei:	Abstract.tex


\addchap{Abstract}
%\addchap[Abstract und Kurzfassung]{Abstract}	% wenn auf selber Seite gewünscht

\selectlanguage{english}



\textbf{\raggedright\Large{xxx Titel der Arbeit Englisch xxx}}


\textbf{xxx Untertitel der Arbeit englisch xxx}

\vspace*{1cm}

xxx Abstract auf Englisch xxx


\selectlanguage{ngerman}

%\clearpage %abstract englisch
%\clearpage

%Datei:	Kurzfassung.tex


\selectlanguage{ngerman}

%\begin{minipage}{\textwidth} 	%wenn auf selber Seite wie das abstract gewüschnt. pagebreaks vorher entfernen
%\chapter*{Kurzfassung}
%\end{minipage}

\addchap{Kurzfassung}


{
\raggedright
\textbf{\Large{xxx Titel der Diplomarbeit Deutsch xxx}}

}

\textbf{xxx Untertitel der Diplomarbeit}

\vspace*{1cm}

xxx Abstract auf Deutsch xxx


 %abstract deutsch
\clearpage


%===================================================================
%Gestaltung Inhaltsverzeichnis
%===================================================================


\setcounter{secnumdepth}{4}	%nummerierte ebenen
\setcounter{tocdepth}{4}	%angeführte ebenen im verzeichnis

\tocsetup{
	title/font = \sffamily\huge\color{farbe1},
	chapter/number/after = .,
	chapter/indent=0mm,
	chapter/number/width=8mm,	
	section/number/after = .,
	section/indent = 8mm,
	section/number/width=10mm,
	subsection/indent=18mm,
	subsection/number/after = .,
	subsection/number/width=10mm,
	subsubsection/indent=28mm,
	subsubsection/number/after = .,
	subsubsection/number/width=13mm
	}
\listofsetup{lot}{
	parskip=0cm,
	title/font = \sffamily\huge\color{farbe1},
	table/number/after = .,
	table/number/width=13mm,	
	}	
\listofsetup{lof}{
	parskip=0cm,
	title/font = \sffamily\huge\color{farbe1},
	figure/number/after = .,	
	figure/number/width=13mm,}

%Nützliche Befehle fürs toc: 

%\addtocontents{toc}{\protect\clearpage} % fügt Zeilenumbruch im Inhaltsverzeichnis nach dem vorangegangenen Eintrag durch. Muss also im Fließtext bzw. zwischen den Überschriften stehen

%\addtocontents{toc}{\protect\enlargethispage{2\baselineskip}} %fügt nach dem vorigen toc-Eintrag den Befehl dazu, dass die Seite im Inhaltsverzeichnis um 2 Zeilen länger wird



{

\renewcommand{\leftmark}{}
\renewcommand{\rightmark}{}

\vspace*{-1.5cm}
\pdfbookmark[0]{Inhalt}{pdfinhalt}
\renewcommand{\contentsname}{Inhalt}
\tableofcontents

}


\clearpage


%Datei:	Danksagung.tex

%Part:	Beginn
%Chapter: 	Danksagung



\selectlanguage{ngerman}
\addchap{Vorwort und Danksagung}
%\thispagestyle{empty}


xxx Optionales Vorwort mit Danksagung... xxx




 %danksagung optional. Wenn nicht gewünscht, dann ausgkommentieren.
\clearpage

%Datei:	Eid.tex

%Part:	Beginn
%Chapter: 	Eidesstattliche Erklärung


\chapter*{Eidesstattliche  Erklärung}



Ich erkläre ehrenwörtlich, dass ich die vorliegende Arbeit selbstständig und
ohne fremde Hilfe verfasst, andere als die angegebenen Quellen nicht benutzt
und die den Quellen wörtlich oder inhaltlich entnommenen Stellen als
solche kenntlich gemacht habe. Die Arbeit wurde bisher in gleicher oder
ähnlicher Form keiner anderen inländischen oder ausländischen Prüfungsbehörde
vorgelegt und auch noch nicht veröffentlicht. Die vorliegende Fassung
entspricht der eingereichten elektronischen Version.

\vspace*{2cm}

\begin{minipage}[b]{0.4\textwidth}
\begin{flushleft}
Graz, am xxx. xxx 201x\newline
\phantom{Zeile}
\end{flushleft}
\end{minipage}
\hfill
\begin{minipage}[b]{0.4\textwidth}
\begin{center}
\vspace*{-1cm}

(Max Mustermann)
\end{center}
\end{minipage}


 %eidesstattliche erkärung
\clearpage

%abkzg.tex

\addchap{Abkürzungsverzeichnis (xxx optional xxx)}


Die folgende Auflistung gibt einen Überblick der wichtigsten in dieser Arbeit vorkommenden Begriffe, Akronyme, Abkürzungen und Formelzeichen:

\bigskip

\begin{center}
{
\renewcommand{\arraystretch}{1.2}
\begin{tabular}{l | l  l}

\multicolumn{1}{l}{} & \textbf{Abkz.} & \textbf{Erklärung} \nzl

A & AHS & Allgemeinbildende Höhere Schulen \nzl


\zentrn{5}{B} & Bac & Bachelor (Fachwissenschaft) \nz
& BHS & Berufsbildende Höhere Schulen (HTL, HLW, HAK) \nz
& BK & Brückenkurs \nz
& BORG & Oberstufenrealgymnasium \nz
& Bsp/Bspe & Beispiel/Beispiele (Übungsaufgaben) \nzl

E & ECTS & Leistungspunkte im European Credit Transfer System \nzl

G & Gym & Gymnasium \nzl

\zentrn{2}{P} & P \& P &  paper and pencil (Fragebogen in Papierform) \nz
& PS&  Lehrveranstaltungstyp Proseminar (siehe UE) \nzl

\zentrn{3}{U} & UE & Lehrveranstaltungstyp Übung \nz
& UGO & Uni Graz Online - Onlinesystem der Uni Graz (LV-Anmeldungen etc.) \nz
& Uni &  Universität (z.\,B. Uni Graz: Karl-Franzens Universität Graz) \nzl

\zentrn{3}{V} & V.I. & Vollständige Induktion \nz 
& VO & Lehrveranstaltungstyp   Vorlesung \nz
& VU & Lehrveranstaltungstyp   Vorlesung mit Übung \nzl

W & WS &  Wintersemester \nzl
\end{tabular}

}
\end{center}







\clearpage


 %abkürzungsverzeichnis optional. Kann auch in den Anhang geschoben werden.

\clearpage
\pagestyle{fancy}
\pagenumbering{arabic} %beginn mit arabischer seitenzählung

%===================================================================================
%Einleitung
%===================================================================================

\chapter{Einleitung}\label{chapter:einleitung}
% kapitel einleitung des ersten hauptteils





xxx optionale einleitung des ersten hauptteils


In dieser Arbeit werden xxx behandelt. \cite{Ableitinger2013kap2} sowie \cite{Aue2013} und \cite{Beutelspacher2011} und die letzte Quelle \cite{BifieStandards}.



%===================================================================================
%Kapitel des Haupteils
%===================================================================================

\chapter{Kapitel 1 des Hauptteils}\label{chapter:xxxname1xxx}
% kapitel 1 des hauptteils




% hauptteil 1: kap 1


\section{Erster Unterpunkt von Kapitel 1}

\begin{table}[htbp]
%
\caption[Beispieltabelle]{Beispieltabelle: Es ist irgendetwas dargestellt... (Angaben in \%)}\label{tab:beipsieltabelle}

\begin{center}
\begin{tabular}{l  l  l}
\obl
\textbf{Links}  & \textbf{Mitte} & \textbf{Rechts} \nzl
A & AHS & Allgemeinbildende Höhere Schulen \nz
Blbla  & und   Text & etwas weniger Text \nz
x &y  & restliches Alphabet \ubl
\end{tabular}
\end{center}
\end{table}



\subsection{Unterunterpunkt Nummer eins}

\begin{boxblau}
Das ist eine Beispielumgebung usw. Das ist eine Beispielumgebung usw. Das ist eine Beispielumgebung usw. Das ist eine Beispielumgebung usw. Das ist eine Beispielumgebung usw. Das ist eine Beispielumgebung usw. Das ist eine Beispielumgebung usw.
\end{boxblau}

\begin{boxblau}[frametitle={Testüberschrift}]
Das ist eine Beispielumgebung usw. Das ist eine Beispielumgebung usw. Das ist eine Beispielumgebung usw. Das ist eine Beispielumgebung usw. Das ist eine Beispielumgebung usw. Das ist eine Beispielumgebung usw. Das ist eine Beispielumgebung usw.
\end{boxblau}

Itemize-Umgebung\mfootnote{Eine Testfußnote über mehrere Zeilen. Eine Testfußnote über mehrere Zeilen.  Eine Testfußnote über mehrere Zeilen. Eine Testfußnote über mehrere Zeilen. Eine Testfußnote über mehrere Zeilen. Eine Testfußnote über mehrere Zeilen. Eine Testfußnote über mehrere Zeilen.}

\begin{itemize}
\item as ist eine Beispielumgebung usw.
\item as ist eine Beispielumgebung usw.
\begin{itemize}
\item as ist eine Beispielumgebung usw.
\item as ist eine Beispielumgebung usw.
\begin{itemize}
\item as ist eine Beispielumgebung usw.
\item as ist eine Beispielumgebung usw.
\begin{itemize}
\item as ist eine Beispielumgebung usw.
\item as ist eine Beispielumgebung usw.
\item  as ist eine Beispielumgebung usw.
\end{itemize}
\item  as ist eine Beispielumgebung usw.
\end{itemize}
\item  as ist eine Beispielumgebung usw.
\end{itemize}
\item  as ist eine Beispielumgebung usw.
\end{itemize}

Enumerate-Umgebung

\begin{enumerate}
\item as ist eine Beispielumgebung usw.
\item as ist eine Beispielumgebung usw.
\begin{enumerate}
\item as ist eine Beispielumgebung usw.
\item as ist eine Beispielumgebung usw.
\begin{enumerate}
\item as ist eine Beispielumgebung usw.
\item as ist eine Beispielumgebung usw.
\begin{enumerate}
\item as ist eine Beispielumgebung usw.
\item as ist eine Beispielumgebung usw.
\item  as ist eine Beispielumgebung usw.
\end{enumerate}
\item  as ist eine Beispielumgebung usw.
\end{enumerate}
\item  as ist eine Beispielumgebung usw.
\end{enumerate}
\item  as ist eine Beispielumgebung usw.
\end{enumerate}


\clearpage
\subsection{Textformatierungen}

Nun folgt eingerückter Text: 

\begin{quote}
Das ist eine Beispielumgebung usw. Das ist eine Beispielumgebung usw. Das ist eine Beispielumgebung usw. Das ist eine Beispielumgebung usw. Das ist eine Beispielumgebung usw. Das ist eine Beispielumgebung usw. Das ist eine Beispielumgebung usw.
\end{quote}

Nun folgt zentrierter Text: 

\begin{center}
Das ist eine Beispielumgebung usw. Das ist eine Beispielumgebung usw. Das ist eine Beispielumgebung usw. Das ist eine Beispielumgebung usw. Das ist eine Beispielumgebung usw. Das ist eine Beispielumgebung usw. Das ist eine Beispielumgebung usw.
\end{center}

Nun folgt linksbündiger Text: 

\begin{flushleft}
Das ist eine Beispielumgebung usw. Das ist eine Beispielumgebung usw. Das ist eine Beispielumgebung usw. Das ist eine Beispielumgebung usw. Das ist eine Beispielumgebung usw. Das ist eine Beispielumgebung usw. Das ist eine Beispielumgebung usw.
\end{flushleft}

Nun folgt rechtsbündiger Text: 

\begin{flushright}
Das ist eine Beispielumgebung usw. Das ist eine Beispielumgebung usw. Das ist eine Beispielumgebung usw. Das ist eine Beispielumgebung usw. Das ist eine Beispielumgebung usw. Das ist eine Beispielumgebung usw. Das ist eine Beispielumgebung usw.
\end{flushright}

Nun folgt fetter Text: oder mit Befehl \textbf{Fett}

\begin{bfseries}
Das ist eine Beispielumgebung usw. Das ist eine Beispielumgebung usw. Das ist eine Beispielumgebung usw. Das ist eine Beispielumgebung usw. Das ist eine Beispielumgebung usw. Das ist eine Beispielumgebung usw. Das ist eine Beispielumgebung usw.
\end{bfseries}

Nun folgt kursiver Text: oder mit Befehl \textit{Kursiv}

\begin{itshape}
Das ist eine Beispielumgebung usw. Das ist eine Beispielumgebung usw. Das ist eine Beispielumgebung usw. Das ist eine Beispielumgebung usw. Das ist eine Beispielumgebung usw. Das ist eine Beispielumgebung usw. Das ist eine Beispielumgebung usw.
\end{itshape}


Ein abschnitt in rmfamily

\rmfamily

Die Webversion des neuen Studienleitfadens für das Studienjahr 2014/15 (Wintersemester und Sommersemester) ist nun verfügbar.
Wir empfehlen allen Erstsemestrigen und Studieninteressierten ihn möglichst genau durchzulesen! Die Druckversion ist gerade in Bearbeitung und wird rechtzeitig zur Erstsemstrigenberatung fertig sein.

\familydefault

Ein abschnitt in sfamily

\sffamily

Die Webversion des neuen Studienleitfadens für das Studienjahr 2014/15 (Wintersemester und Sommersemester) ist nun verfügbar.
Wir empfehlen allen Erstsemestrigen und Studieninteressierten ihn möglichst genau durchzulesen! Die Druckversion ist gerade in Bearbeitung und wird rechtzeitig zur Erstsemstrigenberatung fertig sein.

\familydefault


\subsubsection{Letzte nummerierte Ebene}

\subsubsection{Letzte nummerierte Ebene}

\subsubsection{Letzte nummerierte Ebene}

\clearpage


\section{zweiter Unterpunkt von Kapitel 1}

\begin{figure}[hbtp]
\centering 
\includegraphics[width=0.4\textwidth]{grafiken/uni-graz-logo.pdf} 
\caption[Logo der Uni Graz]{Logo der Uni Graz (ab 2010 oder so). Das vorige Logo hatte keine hellgraue Begrenzungslinie.}
\end{figure}


\section{dritter Unterpunkt von Kapitel 1}









\chapter{Kapitel 2 des Hauptteils}\label{chapter:xxxname2xxx}
% kapitel 1 des hauptteils





\section{Verwendung der Mathe-Satz-etc-Umgebungen}

Verwendung der Umgebungen: (Die anweisungen für das Beweis-Ende sind nur nötig, wenn mit den mdframed-Boxen gearbeitet wird (siehe datei: vorspann: umgebungen). Wird stattdessen  mit den amsthm-Umgebungen gearbeitet, so macht die beweis-umgebung automatisch das symbol.

 
\begin{bsp}[Umformungen aus Pythagoras]\label{bsp:pythagoras}
So wird ein Beispiel formatiert:
\[
a^2 + b^2 = c^2
\]
Daraus ergeben sich die beiden Gleichungen
\begin{align*}
a &= \sqrt{c^2 - b^2} \\
b &= \sqrt{c^2 - a^2}
\end{align*}
durch einfache Umformungen (für $a,b,c>0$).
\end{bsp}

\begin{definition}[rechtwinkeliges Dreieck]\label{defi:rechtwinkeligesdreieck}
Ein Dreieck $\Delta ABC$ heißt rechtwinkelig, wenn es genau einen rechten Winkel (d.\,h. mit \ang{90}) hat. 
\end{definition}

\begin{satz}[Pythagoras]\label{satz:pythagoras}
Es sei $\Delta ABC$ ein Dreieck. Dann sind folgende beiden Aussagen äquivalent:
\begin{enumerate}[label={\alph*)}]
\item  $\Delta ABC$ ist rechtwinkelig.
\item  Es gilt $a^2 + b^2 = c^2$, wobei $a$ und $b$ die Längen der Katheten sind und $c$ die Länge der Hypothenuse ist.
\end{enumerate}
\end{satz}

\begin{beweis}
Der Beweis des Satz \ref{satz:pythagoras} erfolgt durch wildes Gestikulieren. Der Beweis des Satz \ref{satz:pythagoras} erfolgt durch wildes Gestikulieren. Der Beweis des Satz \ref{satz:pythagoras} erfolgt durch wildes Gestikulieren. Der Beweis des Satz \ref{satz:pythagoras} erfolgt durch wildes Gestikulieren. 
\beweisendeinzeile  %auskommentieren, wenn mit amsthm-umgebungen gearbeitet wird
\end{beweis}

\begin{beweis}
Der Beweis des Satz \ref{satz:pythagoras} erfolgt durch wildes Gestikulieren. Der Beweis des Satz \ref{satz:pythagoras} erfolgt durch wildes Gestikulieren. Der Beweis des Satz \ref{satz:pythagoras} erfolgt durch wildes Gestikulieren. Der Beweis des Satz \ref{satz:pythagoras} erfolgt durch wildes Gestikulieren.  Beweis des Satz \ref{satz:pythagoras} erfolgt durch wildes Gestikulieren. 
 
\beweisende  %auskommentieren, wenn mit amsthm-umgebungen gearbeitet wird
\end{beweis}


\begin{beweis}
Der Beweis des Satz \ref{satz:pythagoras} erfolgt durch wildes Gestikulieren. Der Beweis des Satz \ref{satz:pythagoras} erfolgt durch wildes Gestikulieren. Der Beweis des Satz \ref{satz:pythagoras} erfolgt durch wildes Gestikulieren. Der Beweis des Satz \ref{satz:pythagoras} erfolgt durch wildes Gestikulieren. 
\[
c = \sqrt{a^2 + b^2}
\]

\beweisendemathe %auskommentieren, wenn mit amsthm-umgebungen gearbeitet wird
\end{beweis}




\section{Verwendung vom SIunitx-Paket \SI{4 e-3}{kg/s^2}}


Zahlen und Formeln lassen sich  im Text verwenden \SI{25000}{N m}, ebenso  im Mathemodus $x = \SI{123456789}{kg.m/s^2}$ und natürlich auch im abgesetzten Mathe-Modus:
\[
F = \SI{3e-4}{\frac{m}{s^2 kg}}
\]
Nur Zahlen schreibt man als \num{-3 e-7}, nur Einheiten als \si{mol/l}. 

Mathematische Winkel:

\ang{10} \hfill  \ang{12.3} \hfill  \ang{4,5} \hfill  \ang{1;2;3}

Grad-Celsius-Angaben: \SI{100}{\degreeCelsius}

Das Paket bietet noch viele weitere Möglichkeiten für Formatierungen hat noch viele weitere Einheiten vordefiniert.  Beispielsweise kann man Vorsilben für Einheiten verwenden:
\[
\SI{1}{\angstrom} = \SI{e-4}{\micro m} = \SI{e-10}{m}
\]

\section{Verwendung von mhchem und chemfig}

Einige Strukturformeln mit chemfig

\chemfig[][]{-[:30](-[2])-[:330]}
\hfill
\chemfig[][]{C(-[:150]H)(-[:210]H)=C(-[:30]H)(-[:330]H)}

\chemfig[][]{H-Cl} 
\chemfig[][]{H-C(-[2]H)(-[6]H)-C(-[2]Cl)(-[6]H) -H}
\hfill
\chemfig[][]{-[:30]=[:-30]-[:30]-[:-30]}

\begin{center}
\chemfig[][]{\chemabove{O}{\ominus}-[:30](=[:90]O)-[:-30]-[:30]-[:-30]-[:30]-[:-30]-[:30]-[:-30]-[:30]-[:-30]-[:30]-[:-30]-[:30]-[:-30]-[:30]-[:-30]}
\end{center}


\chemfig[][]{HO-P(=[:90]O)(-[:240]O^{-})-[:-30]O--[:60]-{N^+}(-[:270])(-[:90])-[:-30]}
\hfill
\chemfig[][]{ HO-[:30](=[:90]O) -[:-30] -[:30] (-[:-90]OH)(-[:90](-[:150]HO)(=[:30]O) ) -[:-30] -[:30] (=[:90]O) -[:-30]OH }

\chemfig[][]{*6(-=-(-COOH)=-(-HO)=)}
\hfill
\chemfig{C-[:30]
C(-[::-60]H)=[::60]
C(-[::-60]H)-[::60]
C(-[::-60]\chemabove{\lewis{024,O}}{\scriptstyle\ominus})=[::60]
C(-[::-60]H)-[::60]
C(-[::-60]H)=[::60]
C(-[::-60]H)}


Einige Summenformeln mit mhchem: 

\ce{2 H2 + O2 -> 2 H2O}
\hfill
\ce{HAc + H2O <<=> H3O^+ + Ac^-}

\section{Verwendung von tikz und pgf-Plot}


Es folgenden einige Typische Beispiele:

\subsection*{Funktion mit Raster}

\begin{center}
\pgfplotsset{every tick label/.style={inner sep=0pt,font=\scriptsize}}
\pgfplotsset{grid style={dashed, black!80}}
\begin{tikzpicture}
\begin{axis}[
	x=1cm,
	y=1cm,
	grid=major,
    ymin=-2,
    ymax=5,
    xmin=-4,
    xmax=4, 
    xlabel=$x$,
    ylabel=$y$,
    axis on top=true,
    axis x line=middle,
    axis y line=middle,
    axis on top = false,
    ]  
    
    \addplot [very thick, samples=100, domain=-2:4] {(x+1)*(x-2)^2};

\end{axis}
\end{tikzpicture} 
\end{center}

\begin{center}
\pgfplotsset{every tick label/.style={inner sep=0pt,font=\scriptsize}}
\pgfplotsset{grid style={dashed, black!80}}
\begin{tikzpicture}
\begin{axis}[
	x=1cm,
	y=1cm,
	grid=major,
    ymin=-3,
    ymax=7,
    xmin=-7,
    xmax=7, 
    xlabel=$x$,
    ylabel=$y$,
    axis on top=true,
    axis x line=middle,
    axis y line=middle,
    axis on top = false,
    ]  
    
    \addplot [very thick, samples=100, domain=-7:-0.01] {2/x + 1};	%linker ast
     \addplot [very thick, samples=100, domain=0.1:7] {2/x + 1}; %rechter ast
     \addplot [very thick, only marks, samples=2, domain=1:2 ] {2/x + 1};   
\end{axis}	 %zwei markierte Punkte
\end{tikzpicture} 
\end{center}


\clearpage
\subsection*{Exponentialfunktion}

\begin{center}
\pgfplotsset{every tick label/.style={inner sep=0pt,font=\scriptsize}}
\pgfplotsset{grid style={dashed, black!80}}
\begin{tikzpicture}
\begin{axis}[
	%height=9cm,
	x=1cm,
	y=1cm,
	%grid=major,
    ymin=-3,
    ymax=7,
    xmin=-5,
    xmax=5, 
    ticks=none,
    legend pos= north west,
    xlabel=$x$,
    ylabel=$y$,
    axis on top=true,
    axis x line=middle,
    axis y line=middle,
    axis on top = false,
    ]  
    
    \addplot [very thick, dashed, samples=100, domain=-9:5] {2*1.7^x};
    \addlegendentry{$f(x) = 2 \cdot \num{1,7}^x$}
     \addplot [very thick, samples=100, domain=-9:5] {exp(x)};
    \addlegendentry{$g(x)= \ee^x$}
\end{axis}
\end{tikzpicture} 
\end{center}


\subsection*{Mit Hilfslinien}

 \begin{center}
\pgfplotsset{every tick label/.style={inner sep=0pt,font=\scriptsize}}
\begin{tikzpicture}
\begin{axis}[
	%height=9cm,
   x=0.5cm,
   y=2cm,
    ymin=-0.5,
    ymax=2.5,
    xmin=-4,
    xmax=24, 
    legend pos= south east,
    xlabel={$x$ (km waagrechte Entfernung vom Startpunkt)},
    ylabel=$f(x)$ (Flughöhe in km),
    axis on top=true,
    axis x line=middle,
    axis y line=middle
    ]  
    
    \addplot [very thick, samples=200,  domain=0:20] {-x^3/2000 + 3*x^2/200};
    \addplot [thick, dashed, color=red, samples=20,  domain=20:24] {-x^3/2000 + 3*x^2/200};
    \addplot [thick, dashed, color=red,  samples=20, domain=-4:0] {-x^3/2000 + 3*x^2/200};
   
	\draw [dotted, thick](axis cs: 0,2) -- (axis cs: 22,2);
	\draw [dotted, thick](axis cs: 20,0) -- (axis cs: 20,2);
	\draw [dotted, thick](axis cs: 10,0) -- (axis cs: 10,1);
	\draw [dotted, thick](axis cs: 0,1) -- (axis cs: 10,1);

\end{axis}
\end{tikzpicture} 
\end{center}

\clearpage
\subsection*{Zahlengerade mit tikz}

\begin{center}
\begin{tikzpicture}[line cap=round,line join=round,>=triangle 45,x=1cm,y=1cm]

\draw[<->,color=black] (-7.5,0) -- (7.5,0);

\foreach \x in {-7,-6, ..., 7}
\draw[shift={(\x,0)},color=black] (0pt,2pt) -- (0pt,-2pt) node[below] {\footnotesize $\x$};
\end{tikzpicture}
\end{center}


\subsection*{Komplexe Zahlenebene mit tikz}


\begin{center}
\begin{tikzpicture}[line cap=round,line join=round,>=triangle 45,x=0.75cm,y=0.75cm]

\fill [color = hellgrau!120] (-2,-2) rectangle(1, 2);
\draw [line width = 2pt, color = red](-2,-2) -- (-2,2);
\draw[->,color=black] (-5.,0.0) -- (6,0.0);

\foreach \x in {-5, -4, -3, -2, -1 , 1, 2, 3, 4,  5}
\draw[shift={(\x,0)},color=black] (0pt,2pt) -- (0pt,-2pt) node[below] {\footnotesize $\x$};

\draw[->,color=black] (0.0,-3) -- (0.0,5);
\foreach \y in {-2, -1, 1, 2, 3, 4}
\draw[shift={(0,\y)},color=black] (2pt,0pt) -- (-2pt,0pt) node[left] {\footnotesize $\y i$};
\end{tikzpicture}
\end{center}




\begin{center}
\begin{tikzpicture}[line cap=round,line join=round,>=triangle 45,x=0.75cm,y=0.75cm]

\filldraw[fill=hellgrau!120, draw=black, line width=0.1] (2,-3) circle (3);
\fill [color = white] (2,-3) circle (1);
\fill [color = black] (2,-3) circle (0.125);

\fill[fill=red,fill opacity=.6] (5,5.5) -- (-5,-4.5) -- (-5, -6.5) -- (5,3.5);
\draw[draw=black] (5,5.5) -- (-5,-4.5);

\fill[fill=yellow,fill opacity=.6] (-5,1) rectangle (5,2);
\draw[draw=black] (-5,1) -- (5,1);
\draw[draw=black] (-5,2) -- (5,2);

\draw (2,1.5) node {$K$};

%\draw [line width = 2pt, color = red](-2,-2) -- (-2,2);
\draw[->,color=black] (-5.,0.0) -- (5,0.0);

\foreach \x in {-5, -4, -3, -2, -1 , 1, 2, 3, 4,  5}
\draw[shift={(\x,0)},color=black] (0pt,2pt) -- (0pt,-2pt) node[below] {\footnotesize $\x$};

\draw[->,color=black] (0.0,-5) -- (0.0,5);
\foreach \y in {-6, -5, -4, -3, -2, -1, 1, 2, 3, 4}
\draw[shift={(0,\y)},color=black] (2pt,0pt) -- (-2pt,0pt) node[left] {\footnotesize $\y i$};

\end{tikzpicture}
\end{center}





%\addtocontents{toc}{\protect\clearpage} % würde das nächste kapitel  auf eine neue seite im Inhaltsverzeichnis schieben, da seitenumbruch nach dem letzten Inhalts-Verzeichnis-Eintrag gemacht wird

%hier folgenden dann die weiteren Kapitel
%einfach selbst tex-Dateien machen und die kapitel in diese tex-Dateien schreiben (vgl. oben)

\clearpage

%===================================================================================
%Schluss:
%===================================================================================

\chapter{Zusammenfassung}
% kapitel zusammenfassung

xxx  Einleitung des Resümees

\section{Unterkapitel des Resümees}

xxx 

\section{Nächstes Unterkapitel des  Resümees} 




\chapter{Ausblick}
% ausblick
xxx  Einleitung des Ausblicks (z.B. auf weitere Forschung usw)

\section{Unterkapitel des Ausblicks}

xxx 

\section{Nächstes Unterkapitel des  Resümees} 




\clearpage


%===================================================================================
%Literatur
%===================================================================================

\addcontentsline{toc}{chapter}{Literatur}
\renewcommand{\bibname}{Literatur}
\printbibliography 

\clearpage

%===================================================================================
%Weitere Verzeichnisse
%===================================================================================

\phantomsection %hyperverlinkung des nachfolgenden toc-eintrags
\addcontentsline{toc}{chapter}{Abbildungen}  %eintrag der Überschrift ins Verzeichnis
\renewcommand{\listfigurename}{Abbildungen}
\listoffigures
%alle in \begin{figure} ...  \caption{xxx} \end{figure} angeführten Objekte werden aufgeführt 
\clearpage

\phantomsection 
\addcontentsline{toc}{chapter}{Tabellen}  %eintrag der Überschrift ins Verzeichnis
\renewcommand{\listtablename}{Tabellen}
\listoftables %alle in \begin{table} ...  \caption{xxx} \end{table} angeführten Objekte werden aufgeführt



\clearpage

\appendix %schaltet um auf Anhang-Nummerierung A, B, ...
%===================================================================
\addchap{Anhang} %optionaler Anhang 

%alternative Gliederung ohne %addchap{Anhang}
%\chapter{Erster Teil des Anhangs}
%\chapter{Zweiter Teil des Anhangs}
%wenn das auch nicht passt: siehe paket: appendix
%===================================================================

%anhang:


\addsec{Anhang A: Erster Teil vom Anhang}

\addsec{Anhang B: Zweiter Teil vom Anhang}

\addsec{Anhang C: Dritter Teil vom Anhang}




 


\end{document}