% kapitel 1 des hauptteils




% hauptteil 1: kap 1


\section{Erster Unterpunkt von Kapitel 1}

\begin{table}[htbp]
%
\caption[Beispieltabelle]{Beispieltabelle: Es ist irgendetwas dargestellt... (Angaben in \%)}\label{tab:beipsieltabelle}

\begin{center}
\begin{tabular}{l  l  l}
\obl
\textbf{Links}  & \textbf{Mitte} & \textbf{Rechts} \nzl
A & AHS & Allgemeinbildende Höhere Schulen \nz
Blbla  & und   Text & etwas weniger Text \nz
x &y  & restliches Alphabet \ubl
\end{tabular}
\end{center}
\end{table}



\subsection{Unterunterpunkt Nummer eins}

\begin{boxblau}
Das ist eine Beispielumgebung usw. Das ist eine Beispielumgebung usw. Das ist eine Beispielumgebung usw. Das ist eine Beispielumgebung usw. Das ist eine Beispielumgebung usw. Das ist eine Beispielumgebung usw. Das ist eine Beispielumgebung usw.
\end{boxblau}

\begin{boxblau}[frametitle={Testüberschrift}]
Das ist eine Beispielumgebung usw. Das ist eine Beispielumgebung usw. Das ist eine Beispielumgebung usw. Das ist eine Beispielumgebung usw. Das ist eine Beispielumgebung usw. Das ist eine Beispielumgebung usw. Das ist eine Beispielumgebung usw.
\end{boxblau}

Itemize-Umgebung\mfootnote{Eine Testfußnote über mehrere Zeilen. Eine Testfußnote über mehrere Zeilen.  Eine Testfußnote über mehrere Zeilen. Eine Testfußnote über mehrere Zeilen. Eine Testfußnote über mehrere Zeilen. Eine Testfußnote über mehrere Zeilen. Eine Testfußnote über mehrere Zeilen.}

\begin{itemize}
\item as ist eine Beispielumgebung usw.
\item as ist eine Beispielumgebung usw.
\begin{itemize}
\item as ist eine Beispielumgebung usw.
\item as ist eine Beispielumgebung usw.
\begin{itemize}
\item as ist eine Beispielumgebung usw.
\item as ist eine Beispielumgebung usw.
\begin{itemize}
\item as ist eine Beispielumgebung usw.
\item as ist eine Beispielumgebung usw.
\item  as ist eine Beispielumgebung usw.
\end{itemize}
\item  as ist eine Beispielumgebung usw.
\end{itemize}
\item  as ist eine Beispielumgebung usw.
\end{itemize}
\item  as ist eine Beispielumgebung usw.
\end{itemize}

Enumerate-Umgebung

\begin{enumerate}
\item as ist eine Beispielumgebung usw.
\item as ist eine Beispielumgebung usw.
\begin{enumerate}
\item as ist eine Beispielumgebung usw.
\item as ist eine Beispielumgebung usw.
\begin{enumerate}
\item as ist eine Beispielumgebung usw.
\item as ist eine Beispielumgebung usw.
\begin{enumerate}
\item as ist eine Beispielumgebung usw.
\item as ist eine Beispielumgebung usw.
\item  as ist eine Beispielumgebung usw.
\end{enumerate}
\item  as ist eine Beispielumgebung usw.
\end{enumerate}
\item  as ist eine Beispielumgebung usw.
\end{enumerate}
\item  as ist eine Beispielumgebung usw.
\end{enumerate}


\clearpage
\subsection{Textformatierungen}

Nun folgt eingerückter Text: 

\begin{quote}
Das ist eine Beispielumgebung usw. Das ist eine Beispielumgebung usw. Das ist eine Beispielumgebung usw. Das ist eine Beispielumgebung usw. Das ist eine Beispielumgebung usw. Das ist eine Beispielumgebung usw. Das ist eine Beispielumgebung usw.
\end{quote}

Nun folgt zentrierter Text: 

\begin{center}
Das ist eine Beispielumgebung usw. Das ist eine Beispielumgebung usw. Das ist eine Beispielumgebung usw. Das ist eine Beispielumgebung usw. Das ist eine Beispielumgebung usw. Das ist eine Beispielumgebung usw. Das ist eine Beispielumgebung usw.
\end{center}

Nun folgt linksbündiger Text: 

\begin{flushleft}
Das ist eine Beispielumgebung usw. Das ist eine Beispielumgebung usw. Das ist eine Beispielumgebung usw. Das ist eine Beispielumgebung usw. Das ist eine Beispielumgebung usw. Das ist eine Beispielumgebung usw. Das ist eine Beispielumgebung usw.
\end{flushleft}

Nun folgt rechtsbündiger Text: 

\begin{flushright}
Das ist eine Beispielumgebung usw. Das ist eine Beispielumgebung usw. Das ist eine Beispielumgebung usw. Das ist eine Beispielumgebung usw. Das ist eine Beispielumgebung usw. Das ist eine Beispielumgebung usw. Das ist eine Beispielumgebung usw.
\end{flushright}

Nun folgt fetter Text: oder mit Befehl \textbf{Fett}

\begin{bfseries}
Das ist eine Beispielumgebung usw. Das ist eine Beispielumgebung usw. Das ist eine Beispielumgebung usw. Das ist eine Beispielumgebung usw. Das ist eine Beispielumgebung usw. Das ist eine Beispielumgebung usw. Das ist eine Beispielumgebung usw.
\end{bfseries}

Nun folgt kursiver Text: oder mit Befehl \textit{Kursiv}

\begin{itshape}
Das ist eine Beispielumgebung usw. Das ist eine Beispielumgebung usw. Das ist eine Beispielumgebung usw. Das ist eine Beispielumgebung usw. Das ist eine Beispielumgebung usw. Das ist eine Beispielumgebung usw. Das ist eine Beispielumgebung usw.
\end{itshape}


Ein abschnitt in rmfamily

\rmfamily

Die Webversion des neuen Studienleitfadens für das Studienjahr 2014/15 (Wintersemester und Sommersemester) ist nun verfügbar.
Wir empfehlen allen Erstsemestrigen und Studieninteressierten ihn möglichst genau durchzulesen! Die Druckversion ist gerade in Bearbeitung und wird rechtzeitig zur Erstsemstrigenberatung fertig sein.

\familydefault

Ein abschnitt in sfamily

\sffamily

Die Webversion des neuen Studienleitfadens für das Studienjahr 2014/15 (Wintersemester und Sommersemester) ist nun verfügbar.
Wir empfehlen allen Erstsemestrigen und Studieninteressierten ihn möglichst genau durchzulesen! Die Druckversion ist gerade in Bearbeitung und wird rechtzeitig zur Erstsemstrigenberatung fertig sein.

\familydefault


\subsubsection{Letzte nummerierte Ebene}

\subsubsection{Letzte nummerierte Ebene}

\subsubsection{Letzte nummerierte Ebene}

\clearpage


\section{zweiter Unterpunkt von Kapitel 1}

\begin{figure}[hbtp]
\centering 
\includegraphics[width=0.4\textwidth]{grafiken/uni-graz-logo.pdf} 
\caption[Logo der Uni Graz]{Logo der Uni Graz (ab 2010 oder so). Das vorige Logo hatte keine hellgraue Begrenzungslinie.}
\end{figure}


\section{dritter Unterpunkt von Kapitel 1}







