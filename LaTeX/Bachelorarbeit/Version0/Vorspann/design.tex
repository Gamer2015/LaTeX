%design-einstellungen

%==============================================================================
%Farbdefinitionen: (bei Bedarf ändern) Hinweis: Beim Ausdrucken sind die Farben meist etwas dunkler als am Bildschirm (druckerabhängig)
%==============================================================================

\definecolor{igblaud}{RGB}{0,85,212}
\definecolor{blau1}{RGB}{0, 101, 186}

%Hellblau:
\definecolor{igblauh}{RGB}{42,127,255}
\definecolor{blockbg}{RGB}{214, 228, 255}
\definecolor{igblaubg}{RGB}{214, 228, 255}

\definecolor{igoranged}{RGB}{255,121,13}
\definecolor{igorange}{RGB}{255, 216, 77}
\definecolor{igorangehell}{RGB}{255, 213, 142}

\definecolor{hellgrau}{cmyk}{0, 0, 0, 0.4}

%farbzuweisungen:
\colorlet{farbe1}{blau1}
\colorlet{farbe2}{igblauh}
\colorlet{farbe3}{igorange}
\colorlet{farbe4}{igorangehell}

%==============================================================================
%Überschriftengestaltungen: Je nach Lust und Laune ändern
%==============================================================================

\addtokomafont{sectioning}{\sffamily}

\addtokomafont{chapter}{\mdseries}
\addtokomafont{part}{\color{farbe1}}
\addtokomafont{partentry}{\color{farbe1}}
\addtokomafont{chapter}{\color{farbe1}}
\addtokomafont{section}{\mdseries\color{farbe1}}
\addtokomafont{subsection}{\mdseries\color{farbe2}}


%\renewcommand*{\chapterheadstartvskip}{\vspace*{0mm}} %umdefinieren des abstands vor der chapter-überschrift, wenn gewünscht.

%==============================================================================
%Seitenränder
%==============================================================================

\usepackage[a4paper]{geometry}

\geometry{
		%twoside, %auskommentiert = einseitiges Layout
		bottom=25mm, %abstand text - rand
		top = 30mm, %abstand rand - text
		headsep = 1cm} %abstand kopfzeile - text
		

%bleibt für text: 210-60 = 150 mm

%Ränderanpassung für pdf-Version und Druckversion

\ifthenelse{\boolean{boolweb}}
{\geometry{inner=30mm,
		outer=30mm %symmetrische Ränder in digitalversion
		}} 
{\geometry{inner=35mm, 
		outer=25mm %innerer rand größer wegen bindung
		}}

%seitenabschluss unten nicht zwingend durch leerraumaufüllung auf gleicher höhe:
\raggedbottom

%==============================================================================
%Kopf und Fußzeile -- allgemein
%==============================================================================

\usepackage{fancyhdr}		%Paket für Kopf- und Fußzeile


\pagestyle{fancy}
\fancyhf{} 				%alle Kopf- und Fußzeilenfelder bereinigen

%Felder befüllen:
\fancyhead[R]{\small\color{hellgrau}\thepage}					%Seitenanzahl oben rechts 

%Formatierung Kolumnentitel
\renewcommand{\chaptermark}[1]{%
 \markboth{%
Kap. \thechapter.\ #1
 }{}}% vgl. Kursfolien Design, Seite 19.


\fancyhead[L]{\small\color{hellgrau}\nouppercase\leftmark}		%kapitelname: links oben
%ersetze \leftmark durch \rightmark, um die aktuelle section zu bekommen...


%Linienbreiten:
\renewcommand{\footrulewidth}{0pt} 					% keine Trennlinie unten
\renewcommand{\headrulewidth}{1.5pt} 				% obere Trennlinie
%}


%Umgestaltung Kopfzeilentrennlinie
\renewcommand\headrule
  {\color{hellgrau!60}%										%Farbzuweisung der Linie
  \vspace{2pt}												%vertikaler Abstand
  \hrule height\headrulewidth width\headwidth		%Linie (Höhe, Breite) zeichnen
 }


%Umgestaltung plain-Style (wird z.B. bei Chapter usw. automatisch aufgerufen

\fancypagestyle{plain}{%
\fancyhf{} 					%Felder bereinigen
\fancyhead[R]{\small\color{hellgrau}\thepage}		%Seitenzahl, oben, außen
}



%==============================================================================
%Fußnotengestaltungen
%==============================================================================


\usepackage[bottom, splitrule]{footmisc}
\newcommand{\mfootnote}[1]{\footnote{#1}$^)$}	%meine eigener fußnoten befehl. Verwenden statt \foonote{...}. 
\deffootnote[0mm]{0mm}{0mm}{\textbf{\thefootnotemark)} } %Fußnoten beginnen mit fetter, normaler zahl und Klammer. Wenn nicht gewünscht, dann % davor


%==============================================================================
%Beschriftungen von Abbildungen und Tabellen
%==============================================================================

\usepackage[labelfont=bf]{caption} %fette captions (z.B. Tab. xxx)

\captionsetup[figure]{name=Abb.}
\captionsetup[table]{name=Tab.}

%==============================================================================
%aufzählungen etc
%==============================================================================

%itemize-symbol verschiedener ebenen: quadrat in farbe
\renewcommand{\labelitemi}{\color{farbe2}\rule{1.3ex}{1.3ex}}
\renewcommand{\labelitemii}{\color{farbe3}\rule{1.3ex}{1.3ex}}
\renewcommand{\labelitemiii}{\color{farbe2}\rule{1ex}{1ex}}
\renewcommand{\labelitemiv}{\textbullet}


\setdefaultenum						%Nummerierungssymbole (enumerate etc) ändern
{\color{farbe1}\bfseries i)}			%Ebene 1,
{\color{black}\bfseries a)}			%Ebene 2
{\color{farbe2}\bfseries i.}			%Ebene 3
{\color{black}\bfseries A.}			%Ebene 4



%==============================================================================
%design-befehle
%==============================================================================

\newcommand{\ausz}[1]{\textit{#1}}	%bei bedarf anpassen...
	
