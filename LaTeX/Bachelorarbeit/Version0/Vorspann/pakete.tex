%Pakete




%======================================================
%Standard-Mathe-Pakete
%======================================================

\usepackage{amsmath}
\usepackage{amsfonts}
\usepackage{amssymb}
\usepackage{amsthm}		%theorem-umgebungen etc
\usepackage{polynom}	%für polynomdivisonen

\usepackage{textcomp}		%zusätzliche symbole


%======================================================
%rmfamily-Schriften mit Mathe-Unterstützung
%======================================================


%es folgt eine auflistung von schriften, wenn man mit der latex-standardschrift lmodern für die rmfamily (serifenschrift) nicht zufrieden ist. die jeweils zuletzt geladen schrift wird aktiviert). einfach enstprechendes prozent-zeichen der gewünschten schrift entfernen

%\usepackage[sc]{mathpazo} % palatino mit mathe-schrift, recht breit

%\usepackage{mathptmx}		%klassische times mit mathe-schrift

%\usepackage{concrete}	%recht altbacken, schreibmaschinenähnlich: concrete

%\usepackage[charter]{mathdesign}	%schrift charter mit matheschrift.	

%\usepackage{fourier}  	%utopia mit fourier als matheschrift
%\usepackage[utopia]{mathdesign}		%wie fourier, etwas mehr abstand als fourier in mathe-formeln

%\usepackage{fouriernc}	%schoolbook mit fourier als mathe-schrift, recht altmodisch und niedrig

%========abgestimmtes charter-paket=============
%%siehe microtype-doku: pdf: http://mirror.unl.edu/ctan/macros/latex/contrib/microtype/microtype.pdf
%%und source-code: http://mirrors.xmu.edu.cn/CTAN/macros/latex/contrib/microtype/microtype.dtx
%\usepackage[charter]{mathdesign}	%charter mit matheschrift
% \def\rmdefault{bch} % not scaled
%\usepackage{sourcesanspro}	%sanserif-schrift passend für charter 
%\makeatletter
%\renewcommand{\SourceSansPro@scale}{1.02}
%\makeatother
%========ende abgestimmtes charter-paket=============
  


%======================================================
%sf-family-Schriften: 
%======================================================

%eine auswahl an sanserif-schriften. Wer mit der latex-Standardschrift bei der sfamily nicht zufrieden ist, entfernt einfach das Prozentzeichen bei der gewünschten schrift.

%\usepackage[math]{kurier} %angenehme schrift auch für fließtext mit matheschrift

%\usepackage[scaled]{helvet}	%helvetica, recht breit, kombinierbar mit euler-matheschrift
%\usepackage{euler}		%mathe-schrift zu helvetica

%\usepackage[scaled]{berasans}	%bitstream vera sans, eher breit

%\usepackage[default]{sourcesanspro}	%auch für fließtext, fett ist sehr aufdringlich

%\usepackage{DejaVuSansCondensed} %relativ kompakt

%\usepackage{libertine} %relativ verschnörkeld, schön für überschriften



%======================================================
%%fehlerhaft:
%======================================================

%\usepackage[bitstream-charter]{mathdesign}		%evtl fehler?
%\usepackage[urw-garamond]{mathdesign}		%geht nicht
%\usepackage{MinionPro}	% geht nicht...
%\usepackage{mathmanx}	%geht nicht
%\usepackage{pandora}		% geht nicht



%======================================================
%Hauptschriftfamily auswählen:
%======================================================

%\renewcommand{\familydefault}{\sfdefault}		%sans serif als Standard-Schrift, wenn gewünscht


%======================================================
%Schriften und physiaklische Einheiten
%======================================================

\usepackage[
	detect-all,		%alle umgebungseinstellungen der schriften werden auf die Befehle \SI, \num, \si übertragen. Wenn nicht gewünscht: auskommentieren.
	decimalsymbol=comma,
	exponent-product=\cdot,   %Voreinstellung: \times (also z.B. 2 \times 10^3)
	quotient-mode = fraction, %\num{2 / 3} wird als \frac{2}{3} übersetzt
	%per-mode=fraction,		%in Einheiten Brüche
	]{siunitx}



%\AtBeginDocument{\sisetup{detect-family=false, math-rm=\mathrm, text-rm=\rmfamily}}			%sinnvolle option, wenn fließtext in sffamily ist, die mathe schrift aber nicht. mit dieser option werden einheiten und zahlen der Befehle \SI, \si, \num automatisch der mathematik-schrift angepasst







%======================================================
%Textsatz-Pakete
%======================================================

\usepackage[autostyle,german=guillemets]{csquotes} %nachträgliche umgestaltung von anführungszeichen
\usepackage{paralist} %für compactitem
\usepackage{enumitem} % für [label={\alph*)}] oder [label={\roman*)}] usw. bei normalen umgebungen

\usepackage{setspace} %für Befehl \onehalfspacing
\onehalfspacing %für 1,5-facher Zeilenabstand. 



%======================================================
%Grafiken-Pakete
%======================================================

\usepackage{graphicx}			%einbinden von grafiken
\usepackage[x11names, table]{xcolor}		%ermöglichen von farben
\usepackage{pdfpages} 		%einbinden ganzer pdf-seiten (z.B für Anhang etc)


%tikz und pgf:
\usepackage{pgf}
\usepackage{pgfplots}
\pgfplotsset{compat=1.8}
\usepackage{mathtools}
\usepackage{tikz}
\usetikzlibrary{arrows}	% bei darf weitere bibliotheken laden. einfach  googeln...

\usepackage[framemethod=TikZ]{mdframed}			%Boxen mit Rahmen und Füllfarben

%======================================================
%ChehmiePakete
%======================================================

\usepackage{chemfig}		%strukturformeln zeichen

\renewcommand*\printatom[1]{\ensuremath{\mathsf{#1}}} %schrift umstellen auf sanserif in strukturformeln
%\setchemformula{format=\sffamily}
\setdoublesep{0.25700 em}  % 'Bond Spacing'
\setatomsep{2 em}%{1.78500 em}    % 'Fixed Length'
\setbondoffset{0.18265 em} % 'Margin Width'
\newcommand{\bondwidth}{0.09 em}%{0.06642 em} % 'Line Width'
\setbondstyle{line width = \bondwidth}

\usepackage[version=3]{mhchem}	%für chemische summenformeln


%======================================================
%Tabellen-Pakete
%======================================================

\usepackage{tabularx} 	%tabellen mit fester breite-> tabularx-Umgebung
\usepackage{multirow} 	%verschmelzen von Zellen mittels multirow
\usepackage{array}		%erweiterte funktionen bei TAbellen >{} oder @{}
\usepackage{booktabs}		%befehle für abstände und Linien: \midrule etc


%======================================================
%Pakete für Inhaltsverzeichnisse und Einstellungen...
%======================================================

\usepackage{tocloft}		%Foramtierung Abbildung und Tabellenverzeichnis
\usepackage{interfaces}


%=======================================
%Pakete zur Umschaltung von pdf-Digital-Version und Papierversion (größere Rand innen wegen Bindung)
%=======================================

\usepackage{ifthen}		%erlaubt if-Abfrage (für Druck-Web-Umschaltung): druck: png-grafiken, web:pdf-grafiken

\newboolean{boolweb}
\setboolean{boolweb}{true}		%web-version eingeschalten: true; druck einschalten: false

%Syntax für boolweb:
%	\ifthenelse{\boolean{boolweb}}
%	{dann} 					%web= true-Anweisungen
%	{ansonsten}				%web=false (also druck)-Anweiseungen

