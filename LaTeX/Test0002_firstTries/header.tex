% Source: http://home.sandiego.edu/~dhoffoss/latex/latex-paper-template.tex

%----------------------------------------------------------------------------------------------------

\setlength{\textwidth}{17.2cm}     % if you change this, consider changing
\setlength{\evensidemargin}{-.3cm} % side margins to retain centering
\setlength{\oddsidemargin}{-.3cm}

\setlength{\textheight}{23cm}   % if you change this, consider changing
\setlength{\topmargin}{-2cm}  % top margin to retain centering
\setlength{\headsep}{1.6cm}

%----------------------------------------------------------------------------------------------------

\usepackage{latexsym}
\usepackage{enumerate}
\usepackage{enumitem}
\usepackage{epsfig}
\usepackage{graphicx}
\usepackage{color}
\usepackage{float}
\usepackage{subfigure}
\usepackage{makeidx}
\usepackage{fancyhdr}
\pagestyle{fancy}
\usepackage{lastpage}
\usepackage{url}

% Mathematik
\usepackage{amsthm}
\usepackage{amsmath}
\usepackage{amssymb}
\usepackage{amstext}
\usepackage{mathtools}
% \usepackage{extarrows}
% \usepackage{tikz}
% \usetikzlibrary{graphs}
% \usetikzlibrary[graphs]

% Sonstige
\usepackage{ifthen}

%----------------------------------------------------------------------------------------------------

\DeclareSymbolFont{AMSb}{U}{msb}{m}{n}  
\DeclareMathSymbol{\Sph}{\mathbin}{AMSb}{"53} 
\DeclareMathSymbol{\T}{\mathbin}{AMSb}{"54} 
\DeclareMathSymbol{\K}{\mathbin}{AMSb}{"4B} 
\DeclareMathSymbol{\C}{\mathbin}{AMSb}{"43} 
\DeclareMathSymbol{\R}{\mathbin}{AMSb}{"52} 
\DeclareMathSymbol{\Q}{\mathbin}{AMSb}{"51} 
\DeclareMathSymbol{\Z}{\mathbin}{AMSb}{"5A} 
\DeclareMathSymbol{\N}{\mathbin}{AMSb}{"4E} 

\newcommand{\powerset}{\mathcal{P}}

%----------------------------------------------------------------------------------------------------

\newcounter{counter}[subsection]
\newenvironment{template}[2]{
	\refstepcounter{counter}
  	\vspace{.9cm}
  	\noindent    

  	\textbf {
	    \thesubsection.\thecounter \quad #1  \ifthenelse{\equal{#2}{}}{}{({#2})}
	    \newline
  	}
  	\begingroup\itshape
} {
	\endgroup
  	\par\bigskip  
}
\newenvironment{theorem}[1][]{
	\begin{template}{Satz}{#1}
} {
	\end{template}
} 
\newenvironment{lemma}[1][]{
	\begin{template}{Lemma}{#1}
} {
	\end{template}
} 
\newenvironment{corollary}[1][]{
	\begin{template}{Korrolar}{#1}
} {
	\end{template}
} 
\newenvironment{definition}[1][]{
	\begin{template}{Definition}{#1}
} {
	\end{template}
} 
\newenvironment{proposition}[1][]{
	\begin{template}{Proposition}{#1}
} {
	\end{template}
} 
\newenvironment{remark}{
	\textbf{Remark: }
} {
}






\if true false
\newenvironment{testing}[3][dummy]{
	text1: #1, text2: #2, text3: #3
	\newline
} {
}

\newtheorem{test1}{Test}[counter]
\newtheorem{test2}[test1]{Definition}

\newenvironment{Satz}[3][dummy]{
	text1: #1, text2: #2
	\newline
} {
}

\newtheorem{test1}{Test}[counter]
\newtheorem{test2}[test1]{Definition}
\fi

